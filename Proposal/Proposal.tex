\documentclass[a4paper,11pt,pdftex, parskip]{scrreprt}
\usepackage[pdftex]{graphicx}
\usepackage{ngerman}
\usepackage[utf8]{inputenc}
\usepackage[T1]{fontenc}
\usepackage{lmodern}
\usepackage[numbers]{natbib}
\usepackage[margin=5pt,font=small,labelfont=bf, textfont=it]{caption}
%\usepackage[reals]{layout}
\usepackage[lmargin = 2cm, rmargin = 2cm, tmargin = 2cm]{geometry}
\usepackage{listings}
\lstset{language = Python, breaklines = true}
\lstset{numbers=left, numberstyle=\small, stepnumber=2, numbersep=8pt, frame=single, aboveskip= 10pt, belowskip=10pt}


\begin{document}
%\layout*
\widowpenalty = 10000
\clubpenalty = 10000
\displaywidowpenalty = 10000
\begin{titlepage}

  
    \title{
        \begin{figure}[t]
            \centering
            \includegraphics[scale = 0.125, keepaspectratio]{Logos/wwu_logo.png}
            \hspace{1cm}
            \includegraphics[scale = 0.9, keepaspectratio]{Logos/ifgi_logo.png}
        \end{figure}
        % \begin{figure}[t]
        %     \centering
        %     \begin{minipage}[b]{.4\linewidth}
        %         \includegraphics[scale = 0.1, angle = 0]{Logos/wwu_logo.png}
        %        \end{minipage}
        %     \begin{minipage}[b]{.5\linewidth}
        %      \includegraphics[scale = 0.4, angle = 0 ]{Logos/ifgi_logo.png}
        %     \end{minipage}
          
        %    \end{figure}
           
        \textnormal{ 
            \normalsize \Large 
            Westfälische Wilhelms Universität \\ Institut für Geoinformatik\\
            \vspace{3cm}
           % \LARGE 
            Proposal zu\\}
        \glqq {Videobasiertes Objektracking mit \\ der Discrete Curve Evolution}\grqq
        %Semantische Kompressino von Drohnenvideos mit der Discrete Curve Evolution 
        \vspace{1,5cm}}

   
    \author{ 
        Erstgutachter: Prof. Dr. Reinhard Moratz \\
        Zweitgutachter: Dr. Christian Knoth \\
        Ausgabetermin: tbd. \\
        \vspace{0.75cm}
        Abgabetermin: tbd. \\
        Vorgelegt von: Timo Lietmeyer \\
        Geboren am : 23.05.1999 \\
        E-Mail-Adresse: timolietmeyer@uni-muenster.de \\
        Matrikelnummer: 459 169 \\
        Studiengang: B. Sc.  Geoinformatik
    }
    \date{ }    
\end{titlepage}
\maketitle
%\tableofcontents

% \listoffigures \pagebreak


\setcounter{page}{1}

%\section*{1 Motivation}
{\bfseries \large Motivation\\}
Verkehrsüberwachung mit Kameras ist ein wichtiges Element um flüssigen Verkehr zu ermöglichen. \\
Eine Schwierigkeit bei Videoaufzeichnungen liegt im Datenschutz des einzelnen Autos.  \\
Eine Lösung für die Anonymisierung ist die Benutzung Discrete Curve Evolution (DCE).\\
Im Rahmen dieser Bachelorarbeit soll beispielhaft eine Methode in Python implementiert werden, die eine Darstellung von vereinfachten Objekten ermöglicht. Außerdem soll evaluiert werden, inwieweit die DCE Tracking von Objekten in Videos unterstützen kann.
Die Idee ist, dass das Objekt detektiert, in eine Binärmaske umgewandelt, mit der DCE vereinfacht und wieder in einem Video ausgegeben wird.\\%\section*{2 Methodik}%
{\bfseries \large Methodik\\}
Zum Testen wird Videomaterial von Autobahnen aufgenommen, welches im Rahmen der Bachelorarbeit analysiert wird. Ein beispielhafter Verlauf anhand eines einzelnen Bildes ist in Abb. \ref{Bsp_Dorr} zu sehen. \\
Die folgenden Schritte werden für jeden Frame im Video ausgeführt. 
Als ersten Schritt müssen die zu erkennenden Objekte detektiert werden. Dies kann mit der Schwellwertsegmentierung nach Otsu erfolgen, da die Objekte eindeutig zu erkennen sind \citep{Otsu1979}. Außerdem ist die Schwellwertsegmentierung sehr ressourcenschonend, da kein maschinelles Lernverfahren verwendet wird. Eine andere Methode mit einem maschinellen Lernverfahren wäre die Benutzung von YOLO zur Segmentierung. Beide Verfahren beinhalten das Umwandeln in eine Binärmaske als zweiten Schritt. Diese Binärmaske wird im dritten Schritt in ein Polygon umgewandelt, welches mit der DCE vereinfacht wird. \\
Die DCE berechnet anhand eines Grenzwertes, welche Punkte für die Darstellung einer Form irrelevant sind, sodass diese ohne größeren Informationsverlust entfernt werden können \citep{Barkowsky2000}. Dadurch wird eine bedarfsbezogene Vereinfachung des Polygons ermöglicht. 
Weiterhin kann die Vereinfachung die mit dem DCE Algorithmus erreicht wird, eine Überprüfung der Ergebnisse vereinfachen. Hierfür bietet sich Verwendung einer Ähnlichkeitsfunktion für Polygone an. \\
Eine Ergebnisevaluation ist durch die Ausgabe der vereinfachten Frames  oder des gesamten vereinfachten Videos möglich, bei der vorher die einzelnen Bilder zu einem Video zusammengesetzt wurden. \\

\begin{figure}[ht]
 \vspace{-0.5cm}
    \centering
    \includegraphics*[scale = 0.5, keepaspectratio, trim=2 2 2 2 ]{images/Example_bird.png}
    \caption[Beispielablauf der Segmentierung und DCE aus \citet{Dorr2017}]{Beispielablauf einer Vereinfachung mit der DCE \citep{Dorr2017}.}
    \label{Bsp_Dorr}
\end{figure}

%\section*{3 Ausblick}
{\bfseries \large Ausblick\\}
Wenn das Ergebnis der Anwendung von DCE für Objekttracking erfolgreich ist und die getrackten Objekte erkennbar bleiben, kann eine hardwarenahe Programmierung erfolgen. Diese könnte in C oder C++ gemacht werden, um schnellere Ergebnisse liefern zu können, da die Prozessierungsgeschwindigkeit von Python begrenzt ist.\newline
Durch die hardwarenahe Implementierung der DCE könnte eine Anonymisierung direkt am Aufzeichnungsort stattfinden. Auch ein vereinfachtes Objekttracking ist durch eine erfolgreiche Anwendung der DCE möglich.



\appendix

\bibliographystyle{unsrtnat}
\bibliography{Bibliography}



% Die Weiterentwicklung der integrierten Prozessoren in Drohnen ermöglicht möglicherweise in absehbarer Zeit vor Ort semantische Komprimierung mithilfe der DCE. Durch die Senkung der Datenübertragungsrate kann eine stabilere Funkverbindung, sowie höhere Reichweite ermöglicht werden. 



% Drohnen mit Kameras verbreiten sich immer weiter in Deutschland, wodurch eine stabile Verbindung von Pilot zu Drohne von hoher Wichtigkeit ist \citep{Nehring2021}. \newline
% % Da bei der immer weiter voranschreitenden technischen Entwicklung abzusehen ist, dass die Kameraauflösung bei Drohnen weiter steigt \citep{futuretrends2017}, ist auch eine stärkere Kompression dieser Bilder und Videos nötig, um eine stabile Verbindung weiter zu gewährleisten. Diese Kompression kann semantisch erfolgen, indem nur das übertragen wird, was auch gewünscht ist, um noch geringere Datenraten zu ermöglichen. Außerdem ist das Erreichen einer höheren Reichweite bei der Funkverbindung sehr wünschenswert. \newline


%  Bei dem Beispielvideo ist dies durch die statische Kameraposition in Verbindung mit den sich bewegenden Objekten durch Bewegtsegmentierung der beiden Objekte oder mit anderen Bilderkennungsalgorithmen, wie YOLO (You Only Look Once), möglich \Citep{Plastiras2018}. Dies ermöglicht die kontextabhängige Weiterverarbeitung.\newline
% Weitergehend müssen die detektierten Bildsegmente in eine Binärmaske umgewandelt werden, welche vereinfacht werden kann. Das Umwandeln der segementierten Bildausschnitte in eine Binärmaske kann mithilfe eines Schwellwertverfahrens, welches von Nobuyuki Otsu entwickelt wurde, erfolgen \citep{Otsu1979}. \\


% \begin{itemize}
%     \item Sequenzierung von Objekten kann für bessere (semantische) Kompression mithilfe der Discrete Curve Evaluation (DCE) benutzt werden 
%     \item bessere Reichweite, stabilere Funkverbindungen von Drohne zu Pilot 
%     \item Dadurch können auch private Drohnennutzer von besseren Bedingungen profitieren 
%     \item Knoth:
%         \item Konkreten Use Case anführen (nicht nur Bandbreite erhöhen): Nur das übertragen was auch gesehen werden soll
%         \item Anonymisierung von aufgezeichneten Personen
%         \item kein Problem mit Betreuung
%         \item Was passiert mit den komprimierten Daten? Werden die beim Empfänger wieder zurückentpackt?
% \end{itemize}

% \begin{figure}[ht]
%     \centering
%     \includegraphics[scale = 1, keepaspectratio] {images/detail_screenshot_lkw_yt.png}
%     \includegraphics[scale = 2.25, keepaspectratio] {images/detail_screenshot_car_yt.png}
%     \caption[Ausschnitte aus Abb. \ref{Scr_ges_Vid}, welche bewegenden Objekte darstellen ]{Ausschnitte aus Abb. \ref{Scr_ges_Vid}, welche die beiden bewegenden Objekte darstellen \citep{Metz2022}}
%     \label{Scr_detail_Obj}
% \end{figure}

% \begin{figure}[ht]
%     \centering
%     \includegraphics[scale=0.2, keepaspectratio]{images/screenshot_video_autobahn_yt.png}
%     \caption[Screenshot des zu analysierenden Videos]{Screenshot des zu analysierenden Videos \citep{Metz2022}}
%     \label{Scr_ges_Vid}
% \end{figure}

% \begin{itemize}
%     \item Hier kommt ein Ausblickstichpunkt
%     \item Machbarkeitsstudie der DCE und semantischer Kompression auf Verbesserung der Verbindung zwischen Pilot und Drohne
%     \item Anwendung in C/C++, bzw. hardwarenaher umschreiben, damit schnellere (VorOrt-) Prozessierung/Kompression möglich ist
%     \item praktisches Beispiel entwicklen / eigene Drohnensoftware implementieren
% \end{itemize}

% \begin{itemize}
    
%     \item Anwendung von DCE und semantischer Kompression mithilfe von Bilderkennungsalgorithmen (Yolo, etc.)
%     \item Binärmaske aus segementierten Bildmaterial mit Hilfe von  Schwellwertverfahren von Nobuyuki Otsu (möglw.)
    

% \end{itemize}



% \section*{3 Evaluation}
% \begin{itemize}
%     \item Hier kommt ein Evaluationsstichpunkt
%     \item Anwendung auf von Betreuer zur Verfügung gestelltes Videomaterial
% \end{itemize}

% \begin{figure}
%     \lstinputlisting[firstline=131, lastline=140]{../Code/DCE.py}
%      \caption[testpsei]{Programmcode zur Methode XY}
    
% \end{figure}


\end{document}