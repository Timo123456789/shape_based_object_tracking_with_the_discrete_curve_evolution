%!TEX root = ../thesis.tex
\chapter{Fazit und Ausblick}
\label{ch:conclusion}
\section{Fazit}
{
    Zusammenfassen kann man sagen, dass die durchschnittliche Abweichung pro Polygon nach der Vereinfachung von DCE gering genug ist, um ein Tracking zu ermöglichen. Beim Einsatz der größeren Modelle steigt die Winkelabweichung und SSM an, dies ist aber zu vernachlässigen, da die Klassifizierung der Objekte genauer erfolgt.  \\
	Die Implementierung der YOLO Version in der Variante, dass das Video in einzelne Frames zerlegt wird und diese einzeln analysiert werden, hat sich im Vergleich zur direkten vollständigen Analyse mit YOLO als zu ineffektiv herausgestellt und kann damit verworfen werden.
    }
\section{Ausblick}
{
	Aktuell sind feste Punktgrenzen für die einzelnen Klassen zur Berechnung mit der DCE festgelegt. Dies kann durch einen Wert ersetzt werden, der die Ähnlichkeit des vereinfachten Polygons zum Ursprungspolygon misst. Dadurch wird eine bedarfsbezogene Vereinfachung des Polygons für jede Klasse ermöglicht. Dies wird von \citeauthor{Latecki2003} in \citetitle{Latecki2003} \citep{Latecki2003} näher erläutert. \\
	Außerdem wäre eine weitere Möglichkeit, in den Code weitere Klassen mit individuellen Punktgrenzen einzufügen. Dies könnte für eine größere Abdeckung mehrere Szenarien genutzt werden. Begrenzt wird dieses Vorhaben lediglich durch die Anzahl der 80 Klassen, die YOLO unterscheiden kann. \\
	Da die Prozessierungsgeschwindigkeit von Python begrenzt ist, kann eine effizientere Implementierung in C oder C++ erfolgen, um ein echtzeitfähiges System zu ermöglichen.

}

