%!TEX root = ../thesis.tex
\chapter{Einleitung}
\label{ch:intro}

 \todo{siehe und s. immer gleich benutzen (entweder ausschreiben oder nur s.)}
 \todo{Alle Verweise auf Anhang mit Seitenzahl versehen}
 \todo{entweder DCE Substitutionslimits oder -grenzen schreiben}
\section{Motivation}{ 
	
Durch die Verfügbarkeit von Open-Source Software, die neuronale Lernverfahren verwendet und eine sehr schnelle Objekterkennung auf Videoströmen anbietet, lassen sich viele relevante Anwendungen in der Geoinformatik realisieren, beispielsweise im Bereich der Verkehrsüberwachung. \\
In der Arbeitsgruppe \glqq Theoretische und kognitive Grundlagen der Geoinformatik \grqq{} am Institut für Geoinformatik der Universität Münster ist es geplant mit einem hybriden Ansatz Anwendungsmöglichkeiten solcher Systeme zu verbessern. Dabei soll eine geometrische Analyse Ergebnisse der neuronalen Erkennung verifizieren, um das Risiko von Fehldetektionen zu minimieren. Unsere Hypothese vermutet, dass der Ansatz der Discrete Curve Evolution (DCE) von \citet{Latecki1999a}  eine wichtige Stufe sein kann. Diese Vermutung basiert auf Vorarbeiten von \citet*{Dorr2015} an der Universität Maine.
}



\section{Zielsetzung und Forschungskontext}
{%Forschungskontext Reintext
Der Forschungskontext besteht als Basis aus dem Paper von \citeauthor*{Dorr2015}. Bei \citeauthor*{Barkowsky2000} wird die DCE zur Vereinfachung von geometrischen Strukturen auf Karten verwendet \citep{Barkowsky2000}. \\
Weitere Anwendungsfälle sind die Vereinfachung von Skeletten \citep{Latecki2007}, die automatische Kartierung von Entwässerungsnetzen anhand der Geländetopografie mit der Unterstützung der DCE zur Erkennung relevanter Punkte \citep{ZHENG201517} oder auch der medizinische Bereich. Im letzteren kann die DCE bei der Analyse von MRI (Magnetic Resonance Images) helfen \citep{Supot2007}. Des Weiteren kann die DCE bei Gesten- und Handdetektion mit einem Kinect Sensor eingesetzt werden \citep{Lai2016}. Ein ähnlicher Anwendungsfall zu dieser Arbeit, der sich damit beschäftigt nur die relevante Frames aus einem Video zu extrahieren, um die Frames per Second (\glqq Bilder pro Sekunde\grqq{}, FPS) zu minimieren, wurde von \citeauthor*{Latecki2000a} bereits erprobt \citep{Latecki2000a}. \\


Ziel dieser Arbeit ist es ein prototypisches System zu entwickeln, welches Objekte in Videomaterial detektiert und deren Umrisse vereinfacht. Diese Objektumrisse können dann in ihrer Ähnlichkeit verglichen werden, um einzuschätzen, ob die Objekte trackbar sind. Der Ablauf des Programmes ist folgendermaßen geplant. \\ 
Zur Evaluation wird Videomaterial von Autobahnen aufgenommen, welches im Rahmen der Bachelorarbeit analysiert wird. Ein beispielhafter Verlauf anhand eines einzelnen Bildes ist in Abb. \ref{Bsp_Dorr} zu sehen. \\
Die folgenden Schritte werden für jeden Frame im Video ausgeführt. 
Als ersten Schritt müssen die zu erkennenden Objekte detektiert werden. Dies kann mit der Schwellwertsegmentierung nach Otsu erfolgen, da die Objekte eindeutig zu erkennen sind \citep{Otsu1979}. Außerdem ist die Schwellwertsegmentierung sehr ressourcenschonend, da kein maschinelles Lernverfahren verwendet wird. Eine andere Methode mit einem maschinellen Lernverfahren wäre die Benutzung von YOLO zur Segmentierung. Beide Verfahren beinhalten das Umwandeln in eine Binärmaske als zweiten Schritt. Diese Binärmaske wird im dritten Schritt in ein Polygon umgewandelt, welches mit der DCE vereinfacht wird. Dem Maschinellen Lernverfahren YOLO wird gegenüber der Schwellwertsegmentierung den Vorzug gegeben, da die ersten beiden Schritte Detektierung und das Erzeugen der Binärmaske, bzw. des Umrisses der Objekte, bereits in YOLO integriert sind.\\
Die DCE berechnet anhand eines Grenzwertes, welche Punkte für die Darstellung einer Form irrelevant sind, sodass diese ohne größeren Informationsverlust entfernt werden können \citep{Barkowsky2000}. Dadurch wird eine bedarfsbezogene Vereinfachung, anhand eines festen Grenzwertes, des Polygons ermöglicht. 
Weiterhin kann die Vereinfachung die mit dem DCE Algorithmus erreicht wird, eine Überprüfung der Ergebnisse vereinfachen. Hierfür bietet sich Verwendung eines Formähnlichkeitsmaßes für Polygone an. \\
Die Ergebnisevaluation ist durch eine Auswertung des Formähnlichkeitsmaßes möglich. Wenn dieses Maß den entsprechenden Wert hat, kann beurteilt werden, ob Objekttracking mit der DCE möglich ist. Außerdem ist eine visuelle Beurteilung des Videos, welches als Endprodukt entsteht, möglich.\\}
\begin{figure}[ht]
	\vspace{-0.5cm}
	   \centering
	   \includegraphics*[scale = 0.5, keepaspectratio, trim=2 2 2 2 ]{images/Example_bird.png}
	   \caption[Beispielablauf der Segmentierung und DCE]{Beispielablauf einer Vereinfachung mit der DCE \citep{Dorr2017}.}
	   \label{Bsp_Dorr}
\end{figure}


% \todo{mussn noch ausformuliert werden}
% Forschungskontext:
% \begin{itemize}
% 	\item Vereinfachung von Karten (Geometrische Strukturen) \citep{Barkowsky2000}
% 	\item Skeletvereinfachungsmethode \citep{Latecki2007}
% 	\item Kombinierung DCE und Skeleton Construction Technik \citep{ZHENG201517}
% 	\item Medizinischer Kontext MRI \citep{Supot2007}
% 	\item DCE auf digitale Bilder (Videos ) \citep{Latecki2000}
% 	\item Gesten (Fingertips) und Handdetection basierend auf DCE mit Kinect Sensor \citep{Lai2016}
% 	\item DCE Grundlagen \citep{Latecki1998}
% \end{itemize}
\section{Aufbau der Arbeit} \todo{muss überarbeitet werden (Begründung für weitere TR hier ein?)}
{Diese Bachelorarbeit ist in 6 Kapitel aufgeteilt. \\ 
Der Theoretische Hintergrund wird zuerst beschrieben. Dieser enthält eine Einführung in den YOLO-Algorithmus mit einer kurzen Erläuterung der Grundlagen im Maschinellen Lernen, sowie eine Erklärung der Theorie hinter der Discrete Curve Evolution. Außerdem wird die genutzte Programmiersprache definiert und die externen Bibliotheken beschrieben. Des Weiteren werden weitere vom Programm genutzte Algorithmen erklärt und ein kurzer Meta-Ablauf des Programmes skizziert. \\
Die Implementierung bildet mit der Evaluation den Hauptteil der Arbeit. Ersteres beschreibt den Programmablauf an ausgewählten Codebeispielen und zeigt Änderungen im Rahmen der Implementierung zum Theoretischen Hintergrund auf. Die Evaluation ordnet die Ergebnisse verschiedener Testdurchläufe ein. Die Diskussion benennt Gründe für die gewonnenen Ergebnisse der Evaluation. \\
Nach diesem Abschnitt wird ein Fazit gezogen. Der Ausblick in dem Weiterentwicklungen zum Thema dieser Arbeit erläutert werden, schließt diese Arbeit ab. }




