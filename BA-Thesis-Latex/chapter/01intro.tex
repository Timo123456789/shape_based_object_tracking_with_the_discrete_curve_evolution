%!TEX root = ../thesis.tex
\chapter{Einleitung}
\label{ch:intro}


\section{Motivation}{ 
	%PROPOSAL VERSION \\
	% Verkehrsüberwachung mit Kameras ist ein wichtiges Element um flüssigen Verkehr zu ermöglichen. \\
	% Eine Schwierigkeit bei Videoaufzeichnungen liegt im Datenschutz des einzelnen Autos.  \\
	% Eine Lösung für die Anonymisierung ist die Benutzung der Discrete Curve Evolution (DCE).\\
	% Im Rahmen dieser Bachelorarbeit soll beispielhaft eine Methode in Python implementiert werden, die eine Darstellung von vereinfachten Objekten ermöglicht. Außerdem soll evaluiert werden, inwieweit die DCE Tracking von Objekten in Videos unterstützen kann. \\
	% Die Idee ist, dass das Objekt detektiert, in eine Binärmaske umgewandelt, mit der DCE vereinfacht und wieder in einem Video ausgegeben wird.

	% \begin{list}{-}{Notizen}
	% 	\item Leichtgewichtige, schnelle Methode zur Objektdetektierung entwicklen
	% 	\item in Verbindung mit ML; realtime Verarbeitung YOLO
	% \end{list}
	Objekttracking mit maschinellen Lernen ist heutzutage durch die leichte Verfügbarkeit von hoher Prozessor- und Grafikleistung mit wenig Aufwand möglich. Es gibt jedoch Anwendungsfälle, wo aufgrund der äußeren Gegebenheiten weder die Stromversorgung noch der verfügbare Raum den Einsatz möglich machen. \\
	Ein Anwendungsfall, der in diese Kategorie fällt, wäre Wildlife Monitoring mit kleinen Wildkameras und des automatischen Tagging eines Tieres mit einem GPS Sender. Um hier Fehldetektierungen zu vermeiden, wie zum Beispiel, dass eine Gazelle durch falsche Detektierung als Löwe erkannt wird und diese einen Sender bekommt, kann die Formähnlichkeit genutzt werden. Diese kann eine genauere Unterscheidung aus verschiedenen Blickwinkel ermöglichen, um eine höhere Sicherheit bei der Klassifizierung eines Objektes zu erhalten. \\
	Dies kann auch auf Verkehrsüberwachung angewendet werden, indem beispielsweise Autos, die auf einer Autobahn fahren, getrackt werden, um die Verfolgung dieser Autos im Kamerawinkel zu verbessern. Ein Anwendungsfall könnte die Messung von Geschwindigkeit oder die Detektion riskanter Überholmanöver sein.
}
%\blindtext

\section{Zielsetzung}{ 
Ziel dieser Arbeit ist es ein prototypisches System zu entwickeln, welches Objekte in Videomaterial detektiert und deren Umrisse vereinfacht. Der Ablauf des Programmes ist folgendermaßen geplant. \\ 
Zur Evaluation wird Videomaterial von Autobahnen aufgenommen, welches im Rahmen der Bachelorarbeit analysiert wird. Ein beispielhafter Verlauf anhand eines einzelnen Bildes ist in Abb. \ref{Bsp_Dorr} zu sehen. \\
Die folgenden Schritte werden für jeden Frame im Video ausgeführt. 
Als ersten Schritt müssen die zu erkennenden Objekte detektiert werden. Dies kann mit der Schwellwertsegmentierung nach Otsu erfolgen, da die Objekte eindeutig zu erkennen sind \citep{Otsu1979}. Außerdem ist die Schwellwertsegmentierung sehr ressourcenschonend, da kein maschinelles Lernverfahren verwendet wird. Eine andere Methode mit einem maschinellen Lernverfahren wäre die Benutzung von YOLO zur Segmentierung. Beide Verfahren beinhalten das Umwandeln in eine Binärmaske als zweiten Schritt. Diese Binärmaske wird im dritten Schritt in ein Polygon umgewandelt, welches mit der DCE vereinfacht wird. Dmem Maschinellen Lernverfahren YOLO wird gegenüber der Schwellwertsegmentierung den Vorzug gegeben, da die ersten beiden Schritte Detektierung und das Erzeugen der Binärmaske, bzw. des Umrisses der Objekte, bereits in YOLO integriert sind.\\
Die DCE berechnet anhand eines Grenzwertes, welche Punkte für die Darstellung einer Form irrelevant sind, sodass diese ohne größeren Informationsverlust entfernt werden können \citep{Barkowsky2000}. Dadurch wird eine bedarfsbezogene Vereinfachung, anhand eines festen Grenzwertes, des Polygons ermöglicht. 
Weiterhin kann die Vereinfachung die mit dem DCE Algorithmus erreicht wird, eine Überprüfung der Ergebnisse vereinfachen. Hierfür bietet sich Verwendung eines Formähnlichkeitsmaßes für Polygone an. \\
Die Ergebnisevaluation ist durch eine Auswertung des Formähnlichkeitsmaßes möglich. Wenn dieses Maß den entsprechenden Wert hat, kann beurteilt werden, ob Objekttracking mit der DCE möglich ist. Außerdem ist eine visuelle Beurteilung des Videos, welches als Endprodukt entsteht, möglich.\\}
\begin{figure}[ht]
	\vspace{-0.5cm}
	   \centering
	   \includegraphics*[scale = 0.5, keepaspectratio, trim=2 2 2 2 ]{images/Example_bird.png}
	   \caption[Beispielablauf der Segmentierung und DCE]{Beispielablauf einer Vereinfachung mit der DCE \citep{Dorr2017}.}
	   \label{Bsp_Dorr}
\end{figure}
\section{Aufbau der Arbeit}
{Diese Bachelorarbeit ist in 6 Kapitel aufgeteilt. \\ 
Der Theoretische Hintergrund wird zuerst beschrieben. Dieser enthält eine Einführung in den YOLO-Algorithmus mit einer kurzen Erläuterung der Grundlagen im Maschinellen Lernen, sowie eine Erklärung der Theorie hinter der Discrete Curve Evolution. Außerdem wird die genutzte Programmiersprache definiert und die externen Bibliotheken beschrieben. Des Weiteren werden weitere vom Programm genutzte Algorithmen erklärt und ein kurzer Meta-Ablauf des Programmes skizziert. \\
Die Implementierung bildet mit der Evaluation den Hauptteil der Arbeit. Ersteres beschreibt den Programmablauf an ausgewählten Codebeispielen und zeigt Änderungen im Rahmen der Implementierung zum Theoretischen Hintergrund auf. Die Evaluation ordnet die Ergebnisse verschiedener Testdurchläufe ein. Die Diskussion benennt Gründe für die gewonnenen Ergebnisse der Evaluation. \\
Nach diesem Abschnitt wird ein Fazit gezogen. Der Ausblick in dem Weiterentwicklungen zum Thema dieser Arbeit erläutert werden, schließt diese Arbeit ab. }




