%!TEX root = ../thesis.tex
\chapter{Diskussion}
\label{ch:Diskussion}
{ \todo{überarbeiten alles}
	Die beiden verschiedenen YOLO Implementierungen in der EF und RV Version sind sehr ähnlich im Ergebnis und Aufbau. Es zeigt sich jedoch in der Evaluation , dass die EF Variante länger für die Berechnung benötigt und damit ineffektiver ist (siehe Kap. \ref{sec:Ergebnisse}). Dies liegt daran, dass die EF Variante das Video erst in einzelne Frames zerlegt und erst dann YOLO anwendet. Dadurch ist die Prozessierung mit YOLO ineffektiver aber möglicherweise ressourcenschonender im RAM Verbrauch. \todo{das muss überprüft werden; muss mal in die längeren Evaluationsvideos gucken, wie da die Basisdaten ist} Das Zusammensetzen der einzelnen analysierten Frames zu einem Video verlängert die Prozessierungszeit zusätzlich, dies geschieht jedoch auch in der RV Version. \\
	Die RV Version besitzt den Vorteil, dass durch die direkte Verarbeitung des Videos mit dem sehr effiziente geschriebenen YOLO Algorithmus alle Prozessorkerne mitsamt der Grafikkarte vollständig ausgelastet werden können. Zusätzlich steigt jedoch der RAM Verbrauch stark an, da YOLO dort jedes Frame speichert und analysiert. Dies sorgt für eine Zeitersparnis beim Programm. \\ 
	Die DCE Implementierung ist relativ ineffizient, da der K Wert für jeden Punkt im Polygon neu berechnet werden muss, nachdem ein Punkt entfernt wurde. Dies könnte optimiert werden, indem mehrere Punkte eines Polygons gleichzeitig während einer DCE Iteration entfernt werden. Weitere Möglichkeiten wären effizientere Schleifenstrukturen oder Multithreading bei der Berechnung des K Wertes. \\
	Es ist außerdem zu erkennen, dass unterschiedliche YOLO Modelle nur geringe Auswirkungen auf die Gesamtlaufzeit des Programmes haben. Die DCE hat deutlich höheren Einfluss. \\
	
	Mit einem besseren YOLO Modell steigt die SSM für jede Klasse pro Polygon an, weil die Anzahl der erkannten Objekte steigt. Dies kann durch die größeren Trainingsdaten erklärt werden. Die steigende Abweichung, beispielsweise von ca. 20 über ca. 90. bis zu ca. 114 Grad pro detektiertes Auto kann mit der unterschiedlichen Vereinfachung der DCE begründen. Ein Auto ist im ersten Frame anders vereinfacht worden, als im nächsten Frame, wo möglicherweise andere Punkte zum Polygon hinzugekommen sind oder entfernt wurden. Diese Vereinfachung kann für jedes Polygon anders verlaufen, da die DCE die Relevanz von jedem Punkt in jedem neuen Polygon neu berechnet und YOLO die Umrisse mit kleineren Abweichungen ausgibt.

	Bei Objekten, die bei der DCE Vereinfachung als \glqq other Object\grqq{} behandelt werden, können die deutlich höheren SSMs durch die deutlich höheren Punktzahlen, auf die die DCE am Ende reduziert, erklärt werden. Diese führt zu einer steigenden Gesamtwinkelsumme pro Polygon, welches auch zu einer höheren Abweichung der Gesamtwinkelsumme im nächsten Frame führt. Dies kann die gleichen Ursachen wie im obigen Abschnitt haben und ist inbesondere in Kap. \ref{ev:shiptracking} zu erkennen. \\
	
	Einen Einfluss auf absolute Anzahl von Autos können LKW, die Autos transportieren, haben. Einer dieser Autotransporter ist im Testdatensatz zu sehen, was zu einer Erhöhung der absoluten Anzahl von Autos führt, als dieser teilweise aus dem Bild fährt, wodurch nur noch die auf dem Anhänger stehenden Autos von YOLO detektiert werden. Weitere falsche Detektierungen können durch Schilder und andere Objekte verursacht werden. \\

	Wenn man die Ergebnisse aus der Sicht der exakten Berechnungsmöglichkeiten heutiger Computer beurteilt, müsste die Abweichung aus gegen 0 konvergierende Werten bestehen. Durch die hohen Punktunterschiede zwischen den einzelnen Klassen und nicht exaktes Umriss- und Klassifizierungstracking von YOLO ist der leichte Anstieg der SSM erklärbar. Das Tracking von YOLO wechselt zwischen der am besten passenden Klassifizierung, obwohl das Objekt intuitiv vom Betrachter beurteilt auch weiterhin der wirklichen, real übereinstimmenden vorher erkannten Klasse entspricht. 

	\begin{itemize}
		\item warum kann man das alles nicht direkt mit YOLO machen? Rechenleistung ist doch eigentlich genug vorhanden (auch auf Drohnen bzw. Wildkameras)
	\end{itemize}

	\section{Einordnung der Evaluationsergebnisse}
	{
		Bei den allgemeinen Daten, die der Evaluation (s. Kap. \ref{subsec:allgErkenntis}) beschrieben wurden, ist zu erkennen, dass die Zahl der Punkte vor DCE minimal steigt, da auch die Zahl der Polygone ansteigt (s. Tabelle \ref{tab:YOLO8_A10s}). Dies kann daran liegen, dass bei den größeren YOLO Modellen die Detektion und Klassifizierung von Objekten genauer und häufiger erfolgt. Die steigende Punktanzahl nach der DCE Vereinfachung kann ebenfalls mit den größeren YOLO Modellen erklärt werden, da diese auch zu einer höheren Zahl von erkannten Polygonen führen. Aus diesem Grund korreliert die Zahl der Punkte nach der DCE Vereinfachung mit der Zahl der erkannten Polygone (bzw. detektierter Objekte).\\
		Die Anzahl der Punkte nach der DCE Vereinfachung bleibt relativ konstant zwischen den Modellen 8m und 8x, da 8m recht zuverlässig alle Objekte im Video detektiert. Dadurch, dass diese Objekte als Polygone danach von der DCE vereinfacht werden, bleibt die Anzahl der Punkte nach der DCE Vereinfachung recht konstant. Die Anzahl der verglichenen Winkel steigt hingegen deutlich schneller an als die Punktanzahl, da die Polygone permutiert werden. \\
		Bei der SSM ist zu sehen, dass die Zahl der Polygone von 8n zu 8m stark ansteigt, weil mehr Objekte detektiert werden. Im Vergleich von 8m zu 8x bleibt die Zahl der Polygone relativ  konstant, weil beide Modelle alle Objekte detektieren, bzw. die Steigerung zwischen 8m und 8x ist sehr gering, weil im Testdatensatz selbst nicht mehr Objekte enthalten sind. \\
		Die Laufzeit des Programmes unterscheidet sich zwischen den Implementierungen, da YOLO bei direkter Verarbeitung den Vorteil der effizienten Implementierung und vollständigen CPU Auslastung nutzen kann. Die EV Version ist ineffektiv implementiert, weil YOLO jedes Bild einzeln analysiert; das Programm auf dieses Einzelbild DCE anwendet und wieder mit dem nächsten Frame von vorne anfängt. Dieser Unterschied ist insbesondere bei der Verwendung der besser trainierten YOLO Modelle zu erkennen, da dort die Analyse mit YOLO mehr Zeit und Ressourcen benötigt. \\
		Außerdem ist auffällig, dass in den Testdatensätzen Züge detektiert werden. Diese Fehldetektion von LKW kann mit besseren Modellen stark verringert und nahezu eliminiert werden, wie auch in Tabelle \ref{tab:YOLO8_A10s_SSM} zu sehen. Die Anzahl der detektierten Züge entspricht ungefähr der Anzahl der im nächsthöheren Modell mehr detektierten LKW. \\

		Wenn man nun die SSM beurteilt, ist zu erkennen, dass die absolute SSM mit der Verwendung des größeren YOLO Modell steigt, weil mehr Objekte der jeweiligen Klasse richtig detektiert und erkannt werden. Da die SSM pro Frame und Klasse direkt von dem Absolutwert abhängt und weniger Falschdetektionen diesen Wert verzerren, steigt dieser Wert ebenfalls bei Verwendung der besseren YOLO Modelle. Dieser Effekt ist auch bei der absoluten Anzahl detektierter Objekte (bspw. Autos) zu erkennen (s. Tab. \ref{tab:YOLO8_A10s_SSM}). \\
		Bei der Klasse LKW verdoppelt sich die absolute SSM, weil die Punktzahl, auf die DCE diese Objekte reduziert, deutlich höher ist (11) als die von der Klasse Autos (8). Da die absolute SSM steigt, wachsen auch hier die anderen Werte, wie SSM pro Frame und Klasse LKW und SSM pro detektiertem LKW. Bei der absoluten Anzahl der Klasse Zug ist hingegen die Verringerung auf 1 durch die Verwendung der besseren YOLO Modelle zu erklären, da bei 8x 1 LKW falsch detektiert wird. Außerdem wird bei 8x kein LKW als Bus mehr (falsch) detektiert. 
		Aus den obigen Gründen folgt, dass ein bessere trainiertes Modell eine bessere Erkennungsleistung bei längerer Berechnungsdauer hat.  \\

		Bei den weiteren Testfällen ist beim Schiffstracking Datensatz zu sehen, dass die Punktanzahlen vor und nach der DCE Berechnung in etwa um die Videolänge ansteigen. Auch hier ist der Anstieg der Anzahl der verglichenen Winkel bei der SSM Berechnung exponentiell aufgrund der Polygonpermutation. Der Anstieg der Prozessierungszeit entspricht ebenfalls ungefähr dem Unterschied der Sekundenzahl der Testdatensätze (1 Sekunde und 21 Sekunden, Anstiegsfaktor 22 bei Prozessierungszeit). Recht kurz ist hingegen die Berechnungszeit der DCE, die dennoch zwischen den Testdatensätzen ansteigt. Dieser Anstieg liegt an der steigenden Polygonanzahl, ist aber sehr gering, weil allgemein bei diesen Testdaten wenige Polygone vereinfacht werden müssen. \\
		Die Minderung der SSM pro Frame und Klasse Boot hängt mit der nicht so stark ansteigenden absoluten SSM im Gegensatz zur stark steigenden absoluten Anzahl detektierter Boote  zusammen. Die Anzahl der detektierten Boote steigt beim 21-sekündigen Testdatensatz aufgrund der Länge an.\\
		Aus diesem Testfall gefolgert werden, dass sich der Anwendungszweck des formbasierten Objekttrackings mit der DCE nicht nur auf den angedachten Anwendungsfall Verkehrstracking beschränkt, sondern auch andere Anwendungsfälle, wie Schiffs- oder Flugzeugtracking, möglich sind. Damit ist der in Kap. \ref{sec:Aufbau_Arbeit} genannte Grund für den Testfall erfüllt. Die einzige Einschränkung hier ist die Anzahl der Klassen, die von YOLO unterschieden werden können.\\

		Bei dem Testfall, der geringe und hohe DCE Substitionslimits abdeckt, ist Folgendes aufgefallen. Die Anzahl der Punkte vor dem DCE Durchlauf ist bei allen 3 Vergleichstestdatensätzen exakt gleich, weil das Quellvideo und das YOLO Modell übereinstimmen. Außerdem ist YOLO ein deterministischer Algorithmus, der bei gleichen Einstellungsparametern und gleichem Modell, stets das exakt gleiche Ergebnis berechnet. Dies gilt auch für die Anzahlen der erkannten und verglichenen Polygone, sowie bei der absoluten Anzahl detektierte Objekte (und dieser Anzahl der Objekte pro Frame), die aus diesen Gründen exakt gleich sind bei allen 3 Vergleichstestdatensätzen. \\
		Die Anzahl der Punkte nach der DCE Berechnung ist unterschiedlich, weil DCE durch die verschiedenen Substitionslimits auf die Polygone auf verschiedenen Punktgrenzen reduziert. Dies ist der Fall, weil durch die Reduktion der Punkte durch DCE, je nach Einstellungen, mehr oder weniger Punkte die Polygon repräsentieren. Dieser Effekt betrifft auch die Zahl der verglichenen Winkel bei der SSM Berechnung, die jedoch exponentiell wegen der Permutation ansteigt. Von der Punktanzahl nach der DCE Vereinfachung ist die Gesamtwinkelsumme ebenfalls abhängig, deshalb steigt diese stark an. \\
		Die Dauer der Prozessierung sinkt, weil der Berechnungsaufwand für die DCE Vereinfachung nicht mehr so hoch ist. \\
		Da die weiteren SSM Werte von dem absoluten SSM Wert und dieser von der Punktanzahl nach der DCE abhängig ist, sind dort ähnliche Steigerungen zu erkennen. \\
		Die absolute Anzahl detektierter Autos und LKW, bzw. die Anzahl dieser Objekte pro Frame, bleibt über alle 3 Testdatensätze gleich, weil der YOLO Algorithmus, wie oben erläutert, deterministisch ist. \\
		An diesem Testfall konnte gut erläutert werden, welche Änderungen an den Einstellungen die Ergebnisse beim Objekttracking beeinflussen. Es konnte herausgefunden werden, dass der DCE Algorithmus mit optimierten Substitionslimits eine bessere Leistung erzielt. \\


		




	}

	Evaluation Stichpunkte
	\begin{itemize}
		
		
	
		
		
		\item Gleiche DCE Substitionslimits
		\begin{itemize}
			\item Anstieg der Anzahl Punkte nach DCE bei geringen gleichen Limits ist durch das Anheben mancher Substitutionslimits im VGL zur Referenz zu erklären; erklärt auch großen Anstieg bei hohen gleichen Limits
			\item Anzahl vergl. Winkel korreliert mit Anzahl Punkte nach DCE -> Deshalb die Schwankungen bei geringen und hohen Limits im Vergleich zur Referenz
			\item gilt auch für Gesamtwinkelsumme
			\item Erkannte und verglichene Polygone bleiben gleich (gleiches Video, gleicher deterministischer Algorithmus)
			\item Prozessierungszeit bei geringen gleichen Limits steigt an, weil DCE länger rechnen muss: bei hohen gleichen Limits ebenfalls marginaler Anstieg, aber fast gleich; weil sich die Erhöhung über alle Klassen hinweg im Vergleich zur Referenz ausgleicht
			\item XXXXXXXXXXXXXXXXXXXXXXXXXXXXXXXXXXXXXXXXXXXXXXX
			\item abs. SSM korreliert mit Winkel/Punktanzahl und SSM pro Fr und Kl (bzw. pro detektiertes Auto) hängen von abs. SSM ab
			\item absolute Anz. detektierter Autos (auch detektierte LKW) bleibt gleich weil gleiches Video; deterministischer Algorithmus etc.
			\item kein Unterschied der SSM LKW zwischen Referenz und geringen gleichen Limits, weil geringes Limit mit 11 Punkten exakt der Referenz entspricht
	
		\end{itemize}

		\item langer Testdatensatz
		\begin{itemize}
			\item Zahl von Punkten/Winkeln nach und vor DCE korreliert mit Länge der Videos (gilt auch für Anz. vergl. Winkel, Gesamtwinkelsumme, erkannte Polygone, und verglichene Polygone)
			\item Prozessierungszeit weicht davon ab, wegen Videolänge einfach. Ist bei 30sekündigem Testdatensatz länger; Anstieg ist im Kontext von den 1s vlg zu 10s vgl. zu 30 Sekunden auch recht linear (liegt an gleicher Hardware und gleichen Einstellungen, sowie gleicher Codeversion)
			\item XXXXXXXXXXXXXXXXXXXXXXXXXXXXXXXXXXXXXXXXXXXXXXX
			\item absolute SSM steigt aufgrund von längerem Video (= mehr Objekte = höhere Differenzen); deshalb steigt auch die SSM pro Fr. und Kl. bei 1s vs 10s
			\item Da aber bei 30s die absolute Anzahl massiver ansteigt als die absolute SSM sinkt die SSM pro Fr und Kl wieder bei 30s
			\item absolute Anzahl Autos steigt nicht so massiv wie die absolute Anzahl detektierter LKW an -> Anzahl Autos pro Frame sinkt ein wenig, Anzahl LKW pro Frame steigt ein wenig (über alle 3 Videos)
		\end{itemize}
	\end{itemize}



}
% {
% 	\begin{itemize}
		
% 		\item höhere SSMs bei als 'andere Objekte' deklarierte Polygone, da die Punktzahlen auf die diese reduziert werden höher ist, was zu einer höheren Gesamtwinkelsumme pro Polygon führt und damit die Abweichung zur nächsten Gesamtwinkelsumme im nächsten Frame steigt
% 		\item LKW werden aufgrund der langen Form mit recht monotonem Aussehen als Zug oder Bus Fehldetektiert
% 		\item LKW, die Autos transportieren können die absolute Anzahl der PKW stark verfälschen
% 		\item Falsche Detektierungen von Yolo werden durch Schilder und andere Objekte verursacht
% 	\end{itemize}

% }


