%!TEX root = ../thesis.tex
\chapter{Diskussion}
\label{ch:Diskussion}
{ 	
	Die Diskussion ist in zwei Teile aufgeteilt. Im ersten Teil werden die in Kap. \ref{subsec:allgErkenntis} beschriebenen Ergebnisse aufgegriffen, eingeordnet und begründet und im zweiten Teil findet eine Gesamteinordnung der Ergebnisse statt.

	\section{Einordnung der Evaluationsergebnisse}
	{
		Bei den allgemeinen Daten, die der Evaluation (s. Kap. \ref{subsec:allgErkenntis}) beschrieben wurden, ist zu erkennen, dass die Zahl der Punkte vor DCE minimal steigt, da auch die Zahl der Polygone ansteigt (s. Tabelle \ref{tab:YOLO8_A10s}, S. \pageref{tab:YOLO8_A10s}). Dies kann daran liegen, dass bei den größeren YOLO Modellen die Detektion und Klassifizierung von Objekten genauer und häufiger erfolgt. Die steigende Punktanzahl nach der DCE Vereinfachung kann ebenfalls mit den größeren YOLO Modellen erklärt werden, da diese auch zu einer höheren Zahl von erkannten Polygonen führen. Aus diesem Grund korreliert die Zahl der Punkte nach der DCE Vereinfachung mit der Zahl der erkannten Polygone (bzw. detektierter Objekte).\\
		Die Anzahl der Punkte nach der DCE Vereinfachung bleibt relativ konstant zwischen den Modellen 8m und 8x, da 8m recht zuverlässig alle Objekte im Video detektiert. Dadurch, dass diese Objekte als Polygone danach von der DCE vereinfacht werden, bleibt die Anzahl der Punkte nach der DCE Vereinfachung recht konstant. Die Anzahl der verglichenen Winkel steigt hingegen deutlich schneller an als die Punktanzahl, da die Polygone permutiert werden. \\
		Bei der SSM ist zu sehen, dass die Zahl der Polygone von 8n zu 8m stark ansteigt, weil mehr Objekte detektiert werden. Im Vergleich von 8m zu 8x bleibt die Zahl der Polygone relativ  konstant, weil beide Modelle alle Objekte detektieren, bzw. die Steigerung zwischen 8m und 8x ist sehr gering, weil im Testdatensatz selbst nicht mehr Objekte enthalten sind. \\
		Die Laufzeit des Programmes unterscheidet sich zwischen den Implementierungen, da YOLO bei direkter Verarbeitung den Vorteil der effizienten Implementierung und vollständigen CPU Auslastung nutzen kann. Die EF Version ist ineffektiv implementiert, weil YOLO jedes Bild einzeln analysiert; das Programm auf dieses Einzelbild DCE anwendet und wieder mit dem nächsten Frame von vorne anfängt. Dieser Unterschied ist insbesondere bei der Verwendung der besser trainierten YOLO Modelle zu erkennen, da dort die Analyse mit YOLO mehr Zeit und Ressourcen benötigt. \\
		Außerdem ist auffällig, dass in den Testdatensätzen Züge detektiert werden. Diese Fehldetektion von LKW kann mit besseren Modellen stark verringert und nahezu eliminiert werden, wie auch in Tabelle \ref{tab:YOLO8_A10s_SSM} (S. \pageref{tab:YOLO8_A10s_SSM}) zu sehen. Die Anzahl der detektierten Züge entspricht ungefähr der Anzahl der im nächsthöheren Modell mehr detektierten LKW. \\

		Wenn man nun die SSM beurteilt, ist zu erkennen, dass die absolute SSM mit der Verwendung des größeren YOLO Modell steigt, weil mehr Objekte der jeweiligen Klasse richtig detektiert und erkannt werden. Da die SSM pro Frame und Klasse direkt von dem Absolutwert abhängt und weniger Falschdetektionen diesen Wert verzerren, steigt dieser Wert ebenfalls bei Verwendung der besseren YOLO Modelle. Dieser Effekt ist auch bei der absoluten Anzahl detektierter Objekte (bspw. Autos) zu erkennen (s. Tab. \ref{tab:YOLO8_A10s_SSM}, S. \pageref{tab:YOLO8_A10s_SSM}). \\
		Bei der Klasse LKW verdoppelt sich die absolute SSM, weil die Punktzahl, auf die DCE diese Objekte reduziert, deutlich höher ist (11) als die von der Klasse Autos (8). Da die absolute SSM steigt, wachsen auch hier die anderen Werte, wie SSM pro Frame und Klasse LKW und SSM pro detektiertem LKW. Bei der absoluten Anzahl der Klasse Zug ist hingegen die Verringerung auf 1 durch die Verwendung der besseren YOLO Modelle zu erklären, da bei 8x 1 LKW falsch detektiert wird. Außerdem wird bei 8x kein LKW als Bus mehr (falsch) detektiert. 
		Aus den obigen Gründen folgt, dass ein bessere trainiertes Modell eine bessere Erkennungsleistung bei längerer Berechnungsdauer hat.  \\

		Bei den weiteren Testfällen ist beim Schiffstracking Datensatz zu sehen, dass die Punktanzahlen vor und nach der DCE Berechnung in etwa um die Videolänge ansteigen. Auch hier ist der Anstieg der Anzahl der verglichenen Winkel bei der SSM Berechnung exponentiell aufgrund der Polygonpermutation. Der Anstieg der Prozessierungszeit entspricht ebenfalls ungefähr dem Unterschied der Sekundenzahl der Testdatensätze (1 Sekunde und 21 Sekunden, Anstiegsfaktor 22 bei Prozessierungszeit). Recht kurz ist hingegen die Berechnungszeit der DCE, die dennoch zwischen den Testdatensätzen ansteigt. Dieser Anstieg liegt an der steigenden Polygonanzahl, ist aber sehr gering, weil allgemein bei diesen Testdaten wenige Polygone vereinfacht werden müssen. \\
	    Die absolute Anzahl der detektierten Boote steigt schneller als die absolute SSM, sodass der Wert der SSM pro Frame und Klasse Boot sinkt. Aufgrund der Länge des 21-sekündigen Testdatensatz steigt die absolute Anzahl detektierter Boote stark an.\\
		Aus diesem Testfall gefolgert werden, dass sich der Anwendungszweck des formbasierten Objekttrackings mit der DCE nicht nur auf den angedachten Anwendungsfall Verkehrstracking beschränkt, sondern auch andere Anwendungsfälle, wie Schiffs- oder Flugzeugtracking, möglich sind. Damit ist der in Kap. \ref{sec:Aufbau_Arbeit} genannte Grund für den Testfall erfüllt. Die einzige Einschränkung hier ist die Anzahl der 80 Klassen, die aktuell von YOLO unterschieden werden können.\\

		Bei dem Testfall, der geringe und hohe DCE Substitionslimits abdeckt, ist Folgendes aufgefallen. Die Anzahl der Punkte vor dem DCE Durchlauf ist bei allen 3 Vergleichstestdatensätzen exakt gleich, weil das Quellvideo und das YOLO Modell übereinstimmen. Außerdem ist YOLO ein deterministischer Algorithmus, der bei gleichen Einstellungsparametern und gleichem Modell, stets das exakt gleiche Ergebnis berechnet. Dies gilt auch für die Anzahlen der erkannten und verglichenen Polygone, sowie bei der absoluten Anzahl detektierte Objekte (und dieser Anzahl der Objekte pro Frame), die aus diesen Gründen exakt gleich sind bei allen 3 Vergleichstestdatensätzen. \\
		Die Anzahl der Punkte nach der DCE Berechnung ist unterschiedlich, weil DCE durch die verschiedenen Substitionslimits die Polygone auf verschiedenen Punktgrenzen reduziert. Dies ist der Fall, weil durch die Reduktion der Punkte durch DCE, je nach Einstellungen, mehr oder weniger Punkte jeweils ein Polygon repräsentiert. Dieser Effekt betrifft auch die Zahl der verglichenen Winkel bei der SSM Berechnung, die jedoch exponentiell wegen der Permutation ansteigt. Von der Punktanzahl nach der DCE Vereinfachung ist die Gesamtwinkelsumme ebenfalls abhängig, deshalb steigt diese stark an. \\
		Die Dauer der Prozessierung sinkt, weil der Berechnungsaufwand für die DCE Vereinfachung nicht mehr so hoch ist. \\
		Da die weiteren SSM Werte von dem absoluten SSM Wert abhängen und dieser von der Punktanzahl nach der DCE abhängig ist, sind dort ähnliche Steigerungen zu erkennen. \\
		Die absolute Anzahl detektierter Autos und LKW, bzw. die Anzahl dieser Objekte pro Frame, bleibt über alle 3 Testdatensätze gleich, weil der YOLO Algorithmus, wie oben erläutert, deterministisch ist. \\
		An diesem Testfall konnte gut erläutert werden, welche Änderungen an den Einstellungen die Ergebnisse beim Objekttracking beeinflussen. Es konnte herausgefunden werden, dass der DCE Algorithmus mit optimierten Substitionslimits, welche bspw. ein Mittelweg zwischen dem Grenzwert und der Berechnungsdauer sind, eine bessere Leistung erzielt.\\

		An dem Testfall, der gleiche DCE Substitionslimits für alle Klassen verwendet, sieht man, dass ein Anstieg der Punktanzahl nach der DCE Berechnung bei den geringen gleichen Limits vorhanden ist. Dies ist durch das Anheben mancher DCE Klassensubstitutionslimits im Vergleich zur Referenz zu erklären. Ähnlich lässt sich der Anstieg der Punktanzahl im Vergleich mit den hohen gleichen Limits interpretieren. Die Schwankungen im Vergleich zur Referenz bei der Anzahl der verglichenen Winkel zur SSM Berechnung bei geringen und hohen DCE Substitionslimits ist ebenfalls mit der Anzahl der Punkte nach dem Durchlauf der DCE zu begründen, da diese Werte korrelieren. Dies gilt auch für die Gesamtwinkelsumme. \\
		Die Anzahl der erkannten und verglichenen Punkte bleibt über alle Testdatensätze bei diesem Testfall gleich, weil das gleiche Video mit dem gleichen YOLO Modell analysiert wurde und nur die DCE Einstellungen geändert wurden. Es ist außerdem zu erkennen, dass die Prozessierungszeit bei den geringen gleichen DCE Substitionslimits ansteigt, da die DCE Berechnung aufwendiger ist. Bei den hohen gleichen Substitionslimits ist der Anstieg nur marginal, da die Erhöhung der DCE Substitionslimits über alle 4 Klassifizierungseinordnungen hinweg im Vergleich zur Referenz die Dauer der DCE Berechnung ausgleicht. \\
		Da auch hier die absolute SSM Zahl mit der Punktanzahl nach der DCE Vereinfachung korreliert und von dieser die SSM pro Frame und Klasse, sowie pro detektiertes Auto, abhängen, ist ein geringer Anstieg bei den hohen gleichen Substitionslimits zu sehen. Es gibt keinen Unterschied zwischen der SSM der Klasse LKW und den von ihr abhängenden Werten im Vergleich von dem Testfall mit den geringen gleichen Limits und der Referenz, weil das Substitionslimit exakt der Referenz mit der Klasse LKW entspricht. Bei der Klasse Auto hingegen sind geringe Schwankungen zu erkennen. Bei Objekten, die nicht den 3 Hauptklassen zugehörig sind, ist voraussichtlich ebenfalls kein Unterschied der SSM zwischen der Referenz und den hohen gleichen Substitionslimits zu erkennen, weil diese Substitionslimits exakt einander entsprechen. Die absolute Anzahl detektierter Objekte bleibt über alle Videos gleich, wegen des deterministischen Algorithmus und der gleichen Testdatensätze, sowie fast gleichen Einstellungen. \\
		Bei diesem Testfall waren die Ergebnisse nicht wie erwartet, dass die SSM weiter als in den vorgesehenen Testfällen gegen 0 geht. Dies liegt voraussichtlich an der Ungenauigkeit von YOLO in Verbindung mit Berechnungsungenauigkeiten der DCE. \\

		Bei den Ergebnissen des langen Testdatensatzes sieht man, dass die Zahl der Punkte nach und vor der DCE Berechnung mit der Länge des Eingabevideos korreliert. Diese Beobachtung gilt auch für die Anzahl der verglichenen Winkel zur SSM Berechnung, der Gesamtwinkelsumme und der Anzahl der erkannten und verglichenen Polygone. Von dieser Folgerung weicht die Prozessierungszeit ab, da die Dauer des Programms mit längerem Eingabevideo steigt. Dieser Anstieg ist im Vergleich von dem 1-sekündigen Testdatensatz über den 10-sekündigen Testdatensatz bis hin zum 30-sekündigen Testdatensatz recht linear, da die gleiche Hardware, gleiche Einstellungen und die gleiche Programmcodeversion genutzt wurden. 
		Der Wert der absoluten SSM steigt aufgrund des längeren Videos, weil mehr Objekte zu höheren aufsummierten Winkeldifferenzen führen. Dies ist auch bei den davon abhängigen Variablen zu sehen und ebenfalls bei der SSM pro Frame und Klasse Auto und LKW, wenn man die beiden Referenzvideos von 1 Sekunden und 10 Sekunden Länge miteinander vergleicht. Falls man hier aber den 30-sekündigen Testdatensatz betrachtet, ist zu erkennen, dass die absolute Anzahl detektierter Objekte schneller ansteigt als der Wert der absoluten SSM. Dies führt bei der Klasse Auto dazu, dass die SSM pro Frame und Klasse Auto minimal beim längeren Video absinkt. Ebenfalls davon betroffen ist die SSM pro detektiertes Auto. \\
		Bei der Klasse LKW hingegen steigt die absolute Anzahl beim längeren Video stärker an als bei der Klasse Auto, sodass bei der SSM pro detektiertes Auto eine Verringerung im Vergleich von der 10-sekündigen Referenz zum 30-sekündigem Testdatensatz entsteht und die SSM pro detektiertem LKW leicht ansteigt. Dieser Effekt betrifft auch die SSM pro Frame und Klasse LKW. Bei der Klasse LKW steigt auch die pro Frame detektierter Anzahl LKW konstant über die steigende Videolänge an, während bei der Klasse Auto dieser Wert erst kurz sinkt, bevor er beim langen Testdatensatz sehr gering ansteigt. \\
		Bei diesem Testfall ist zu sehen, dass die berechneten Werte übertragbar sind auf längere Videos. Damit ist eine Beurteilung möglich, wie das Programm mit noch längeren Videos arbeitet, solange die Begrenzungen der Hardware es ermöglichen. Der erkannte Verlauf der Prozessierungsdauer und des Formähnlichkeitsmaßes ist ein teilweise linearer Anstieg. \\
	}
	\section{Gesamteinordnung}
	{ 
	Die beiden verschiedenen YOLO Implementierungen in der EF und RV Version sind sehr ähnlich im Ergebnis und Aufbau. Es zeigt sich jedoch in der Evaluation, dass die EF Variante länger für die Berechnung benötigt und damit ineffektiver ist (s. Kap. \ref{sec:Ergebnisse}). Dies liegt daran, dass die EF Variante das Video erst in einzelne Frames zerlegt und erst dann YOLO anwendet. Dadurch ist die Prozessierung mit YOLO ineffektiver aber möglicherweise ressourcenschonender im RAM Verbrauch. 
	Das Zusammensetzen der einzelnen analysierten Frames zu einem Video verlängert die Prozessierungszeit zusätzlich, dies geschieht jedoch auch in der RV Version. \\
	Die Implementierung der EF Version hat sich im Vergleich zu der RV Version als zu ineffektiv herausgestellt und kann damit verworfen werden.
	Die RV Version besitzt den Vorteil, dass durch die direkte Verarbeitung des Videos mit der sehr effizient geschriebenen YOLO Implementierung von Ultralytics alle Prozessorkerne mitsamt der Grafikkarte vollständig ausgelastet werden können. Zusätzlich steigt jedoch der RAM Verbrauch stark an, da YOLO dort jedes Frame speichert und analysiert. Dies sorgt für eine Zeitersparnis beim Programm. \\ 
	Die DCE Implementierung ist relativ effizient, da der K Wert nur für die Nachbarpunkte des entfernten Punktes im Polygon neu berechnet werden muss, nachdem der Punkt mit dem geringsten K-Wert entfernt wurde. Der Algorithmus könnte trotzdem optimiert werden, indem mehrere Punkte eines Polygons gleichzeitig während einer DCE Iteration entfernt werden. Hier besteht die Herausforderung darin, die Formähnlichkeit weiter zu erhalten. Weitere Möglichkeiten wären effizientere Schleifenstrukturen oder Multithreading bei der Berechnung des K Wertes. \\
	Es ist außerdem zu erkennen, dass unterschiedliche YOLO Modelle eine hohe Auswirkung auf die Gesamtlaufzeit des Programmes haben. Die DCE Berechnung hat hier nur indirekt einen Einfluss, da diese von der Anzahl der detektierten Objekte abhängt. \\
	
	Mit einem besseren YOLO Modell steigt die SSM für jede Klasse pro Polygon an, weil die Anzahl der erkannten Objekte steigt. Dies kann durch die größeren Trainingsdaten erklärt werden. Die steigende Abweichung pro detektiertes Objekt kann mit der unterschiedlichen Vereinfachung der DCE begründet werden. Ein Auto ist beispielsweise im ersten Frame anders vereinfacht worden, als im nächsten Frame, wo möglicherweise andere Punkte zum Polygon hinzugekommen sind oder entfernt wurden. Diese Vereinfachung kann für jedes Polygon anders verlaufen, da die DCE die Relevanz von jedem Punkt in jedem neuen Polygon berechnet und YOLO die Umrisse mit kleineren Abweichungen ausgibt.

	Bei Objekten, die bei der DCE Vereinfachung als \glqq other Object\grqq{} behandelt werden, können die deutlich höheren SSMs durch die deutlich höheren Punktzahlen, auf die die DCE am Ende reduziert, erklärt werden. Diese führt zu einer steigenden Winkeldifferenz pro Polygon im Vergleich zum nächsten Polygon, welches auch zu einer höheren Abweichung der Gesamtwinkeldifferenzsumme im nächsten Frame führt. Dies kann die gleichen Ursachen wie im obigen Abschnitt haben und ist inbesondere in Kap. \ref{ev:shiptracking}, Schiffstracking, zu erkennen. \\

	Einen Einfluss auf absolute Anzahl von Autos können LKW, die Autos transportieren, haben. Einer dieser Autotransporter ist im Testdatensatz zu sehen, was zu einer Erhöhung der absoluten Anzahl von Autos führt, als dieser teilweise aus dem Bild fährt, wodurch nur noch die auf dem Anhänger stehenden Autos von YOLO detektiert werden. Weitere falsche Detektierungen können durch Schilder und andere Objekte, sowie kleinere Kamerabewegungen, verursacht werden. \\
	Allgemein sind die absoluten Zahlen detektierter Klassen nicht gleichzusetzen mit einem Zählen von bspw. Autos, die durch das Video gefahren sind. Dies ist der Fall, weil dieser Wert durch das Summieren aller erkannten Klassenobjekte aller Frames entsteht, was dazu führt, dass dieser Wert deutlich zu hoch ist.

	Der Einfluss der DCE auf die Berechnungsdauer des Programms war durch die relativ effiziente Implementierung mit NumPy gering. Ein Beitrag hierzu liefert die Berechnung des K-Wertes nur anhand der Nachbarpunkte des entfernten Polygons. Wenn der K-Wert jedes Punktes nicht lokal, sondern global für ein Polygon berechnet werden müsste, würde dies den Berechnungsaufwand vervielfachen.

	Wenn man die Ergebnisse aus der Sicht der exakten Berechnungsmöglichkeiten heutiger Computer beurteilt, müsste die Formähnlichkeitsmesswerte aus gegen 0 konvergierende Werten bestehen. Durch die hohen Punktunterschiede zwischen den einzelnen Klassen und nicht exaktes Umriss- und Klassifizierungstracking von YOLO ist der leichte Anstieg der SSM erklärbar. Das Tracking von YOLO wechselt zwischen der am besten passenden Klassifizierung, obwohl das Objekt intuitiv vom Betrachter beurteilt auch weiterhin der wirklichen, real übereinstimmenden vorher erkannten Klasse entspricht. \\

	YOLO könnte das Objekttracking auch direkt vornehmen. Für die Verknüpfung mit der Discrete Curve Evolution ist dies jedoch nicht nötig, da die Vorprozessierung von YOLO lediglich die Objektdetektion und das Erstellen der Objektumrisse enthält. Wenn man die DCE, die einen sehr viel geringere Anforderungen an die Hardware als YOLO hat, mit einer ressourcenschonenderen Segmentierungsmethode, wie der Schwellwertsegmentierung nach \citeauthor{Otsu1979} \cite{Otsu1979} implementiert, kann man ein sehr schlankes und schnelles System zum Objekttracking erhalten. \\
	Auf der anderen Seite kann wiederum durch die fortschreitende technische Entwicklung der Bedarf an schlanken und schnellen Systemen zum Objekttracking sinken, wenn YOLO auf kompakterer und deutlich preiswerter Hardware genauso leistungsfähig ist, wie auf heutigen leistungsstarken aber teuren Computern. \\

	Ein weiterer Anwendungsfall für die in der Arbeit erprobte Technik wäre Signalkomprimierung von Videoströmen. Dadurch, dass durch die KI und Kamera getrackte Objekte detektiert werden, ist es möglich nur die Umrisse, bzw. Ausschnitte der Boundingboxen vom Aufzeichnungsort (Drohne) zum Betrachter (Fernbedienung mit Bildschirm) zu übertragen, um die Datenübertragung und -verbindung zur Drohne zu verbessern. Aufgrund der kompakteren Datenübertragung kann möglicherweise die Verbindung zur Drohne mit einer geringen Frequenzleistung zur Video- und Bildübertragung erfolgen, wodurch weitere Vorteile entstehen. Dieses Anwendungsbeispiel ist nicht nur auf Drohnen, sondern auch auf andere kamerabasierte Tätigkeiten (bspw. Flugzeuge, Tauchroboter, etc.) übertragbar, bei denen der Betrachter durch eine Funkverbindung mit der Kamera verbunden ist.\\

	Diese Arbeit hat einen Beitrag zur Anonymisierung von Verkehrsteilnehmern beim Verkehrsmonitoring geleistet, und eine Verknüpfung von Künstlicher Intelligenz in Form von YOLO mit der Discrete Curve Evolution erprobt. Im Forschungskontext gesehen, hat diese Arbeit eine Verbindung zwischen hocheffiziente Bildklassifizierungssystemen und schnellen Polygonvereinfachungsverfahren geschaffen. Es konnte ein Teil des Forschungskontextes Verkehrsmonitoring mit einem Teil des Forschungskontextes der Discrete Curve Evolution verbunden werden. Dadurch konnte gezeigt werden, dass die DCE sich auch zur Anwendung auf Videoströmen zum Verkehrsmonitoring in Verbindung mit KI eignet. Wenn man jetzt  die Zählung einzelner Klasse optimiert, ist bspw. eine Zählung aller durch das Video fahrenden Autos auch mit YOLO und der DCE möglich. Dies könnte dann in einem intelligenten Verkehrsleitssytem eingesetzt werden. \\
	Das System außerdem kann durch die Anonymisierung des gesamten Verkehrsteilnehmers als Umriss einen neuen Ansatz darstellen, nicht nur bspw. die Nummernschilder zu verpixeln. Es wird ein ganzheitlicher Ansatz der Anonymisierung erprobt, indem nur der Umriss des Verkehrsteilnehmers insgesamt gezeigt wird.
	Weiterentwicklungen wären auch im Bereich Unfallerkennung mit der DCE möglich, sodass bei einem Unfall, der von einer Verkehrsbeobachtungskamera erfasst wird, nur die essenziellen Bilder übertragen werden. 
	}
}



