%!TEX root = ../thesis.tex
\chapter{Diskussion}
\label{ch:Diskussion}
{ \todo{überarbeiten alles}
	Die beiden verschiedenen YOLO Implementierungen in der EF und RV Version sind sehr ähnlich im Ergebnis und Aufbau. Es zeigt sich jedoch in der Evaluation , dass die EF Variante länger für die Berechnung benötigt und damit ineffektiver ist (siehe Kap. \ref{sec:Ergebnisse}). Dies liegt daran, dass die EF Variante das Video erst in einzelne Frames zerlegt und erst dann YOLO anwendet. Dadurch ist die Prozessierung mit YOLO ineffektiver aber möglicherweise ressourcenschonender im RAM Verbrauch. \todo{das muss überprüft werden; muss mal in die längeren Evaluationsvideos gucken, wie da die Statistik ist} Das Zusammensetzen der einzelnen analysierten Frames zu einem Video verlängert die Prozessierungszeit zusätzlich, dies geschieht jedoch auch in der RV Version. \\
	Die RV Version besitzt den Vorteil, dass durch die direkte Verarbeitung des Videos mit dem sehr effiziente geschriebenen YOLO Algorithmus alle Prozessorkerne mitsamt der Grafikkarte vollständig ausgelastet werden können. Zusätzlich steigt jedoch der RAM Verbrauch stark an, da YOLO dort jedes Frame speichert und analysiert. Dies sorgt für eine Zeitersparnis beim Programm. \\ 
	Die DCE Implementierung ist relativ ineffizient, da der K Wert für jeden Punkt im Polygon neu berechnet werden muss, nachdem ein Punkt entfernt wurde. Dies könnte optimiert werden, indem mehrere Punkte eines Polygons gleichzeitig während einer DCE Iteration entfernt werden. Weitere Möglichkeiten wären effizientere Schleifenstrukturen oder Multithreading bei der Berechnung des K Wertes. \\
	Es ist außerdem zu erkennen, dass unterschiedliche YOLO Modelle nur geringe Auswirkungen auf die Gesamtlaufzeit des Programmes haben. Die DCE hat deutlich höheren Einfluss. \\
	
	Mit einem besseren YOLO Modell steigt die SSM für jede Klasse pro Polygon an, weil die Anzahl der erkannten Objekte steigt. Dies kann durch die größeren Trainingsdaten erklärt werden. Die steigende Abweichung, beispielsweise von ca. 20 über ca. 90. bis zu ca. 114 Grad pro detektiertes Auto kann mit der unterschiedlichen Vereinfachung der DCE begründen. Ein Auto ist im ersten Frame anders vereinfacht worden, als im nächsten Frame, wo möglicherweise andere Punkte zum Polygon hinzugekommen sind oder entfernt wurden. Diese Vereinfachung kann für jedes Polygon anders verlaufen, da die DCE die Relevanz von jedem Punkt in jedem neuen Polygon neu berechnet und YOLO die Umrisse mit kleineren Abweichungen ausgibt.

	Bei Objekten, die bei der DCE Vereinfachung als \glqq other Object\grqq{} behandelt werden, können die deutlich höheren SSMs durch die deutlich höheren Punktzahlen, auf die die DCE am Ende reduziert, erklärt werden. Diese führt zu einer steigenden Gesamtwinkelsumme pro Polygon, welches auch zu einer höheren Abweichung der Gesamtwinkelsumme im nächsten Frame führt. Dies kann die gleichen Ursachen wie im obigen Abschnitt haben und ist inbesondere in Kap. \ref{ev:shiptracking} zu erkennen. \\
	
	Einen Einfluss auf absolute Anzahl von Autos können LKW, die Autos transportieren, haben. Einer dieser Autotransporter ist im Testdatensatz zu sehen, was zu einer Erhöhung der absoluten Anzahl von Autos führt, als dieser teilweise aus dem Bild fährt, wodurch nur noch die auf dem Anhänger stehenden Autos von YOLO detektiert werden. Weitere falsche Detektierungen können durch Schilder und andere Objekte verursacht werden. \\

	Wenn man die Ergebnisse aus der Sicht der exakten Berechnungsmöglichkeiten heutiger Computer beurteilt, müsste die Abweichung aus gegen 0 konvergierende Werten bestehen. Durch die hohen Punktunterschiede zwischen den einzelnen Klassen und nicht exaktes Umriss- und Klassifizierungstracking von YOLO ist der leichte Anstieg der SSM erklärbar. Das Tracking von YOLO wechselt zwischen der am besten passenden Klassifizierung, obwohl das Objekt intuitiv vom Betrachter beurteilt auch weiterhin der wirklichen, real übereinstimmenden vorher erkannten Klasse entspricht. 

	\begin{itemize}
		\item warum kann man das alles nicht direkt mit YOLO machen? Rechenleistung ist doch eigentlich genug vorhanden (auch auf Drohnen bzw. Wildkameras)
	\end{itemize}


	Evaluation Stichpunkte
	\begin{itemize}
		\item A10s Allgemeine Daten
		\begin{itemize}
			\item Zahl der Punkte vor DCE steigt minimal, weil auch die Zahl der erkannten Polygone ansteigt (bzw. bei größeren Modellen YOLO sicherer ist das ein Objekt einer Klasse entspricht, bzw. überhaupt detektiert wird)
			\item Zahl der Punkte nach DCE steigt mit der Zahl der erkannten Polygone (bzw. besseren YOLO Modellen) an, weil beides miteinander korreliert (voneinander abhängt)
			\item Zahl der Punkte nach DCE im Vergleich zwischen 8m und 8x bleibt konstant, weil 8m schon gut genug alle Objekte richtig klassifziert und die Reduktion durch DCE dann alle Werte auf ähnliche Punktzahlen reduziert
			\item Die Anzahl der verglichenen Winkel ist deutlich höher, als die Punktanzahl aufgrund der Permutation der zu vergleichenden Polygone.
			\item XXXXXXXXXXXXXXXXXXXXXXXXXXXXXXXXXXXXXXXXXXXXXXX
			\item Zahl der Polygone steigt von 8n zu 8m stark an, weil mehr Objekte detektiert werden, im Vergleich von 8m zu 8x bleibt die Zahl relativ konstant, weil beide Modelle alle Objekte detektieren, bzw. Steigerung ist nicht mehr so extrem, weil einfach nicht mehr vorhanden ist
			\item Berechnungszeit unterscheidet sich zwischen den Implementierungen, da YOLO bei der direkten Verarbeitung den Vorteil der effizienten Implementierung und vollen CPU Auslastung vollständig nutzen kann; EFV ist einfach zu ineffektiv implementiert, weil YOLO jedes Bild einzeln bearbeitet, stoppt, DCE anwendet, und wieder von vorne beginnt 
			\item Unterschied ist inbesondere bei Verwendung der besseren Modelle zu erkennen, weil diese mehr Rechenzeit von YOLO fordern
			\item Züge werden erkannt weil LKW falsch detektiert werden (kurz Zahlen anschauen zur Untermauerung (zug anzahl entspricht ungefähr differnz zu LKW Zahl bei größerem Modell))
		\end{itemize}
		\item Beurteilung des Formähnlichkeitsmesswertes
		\begin{itemize}
			\item Absolute SSM steigt, weil mehr Autos/Objekte der jeweiligen Klasse erkannt werden; SSM pro Frame und Auto hängt davon ab und steigt deshalb ebenfalls; SSM pro Auto sinkt, weil mehr Autos richtig detektiert werden und damit weniger Falschdetektionen das Ergebnis verzerren; absoloute Anz. detektierter Autos steigt weil besseres Modell und damit bessere Erkennung
			\item Verdoppelung der Absoluten SSM bei LKW so hoch, weil LKW mehr Punkte haben, bei 8m und 8x nicht mehr so starke Steigung
			\item Da alle anderen Werte vom SSM abhängen ist dies auch bei den anderen Werten der Fall
			\item Bei dem Zug/Bus ist die Verringerung durch die besseren Modelle zu erklären, da die Modelle genauer und besser Autos/LKW detektieren, ist insbesondere bei 8x zu sehen wo nur noch 1 Zug falsch detektiert wird und kein Bus mehr (Folgerung besser trainiertes Modell gleich bessere Erkennungsleistung -> Logisch)
		\end{itemize}
		\item Weitere Testfälle
		\item shiptracking
		\begin{itemize}
			\item Faktorsteigung steigt grob gesehen um das 21 fache (also in etwa die Videolänge)
			\item Anzahl verglichener Winkel steigt exponentiell an, wegen der Permutation der Polygone
			\item Steigerung der Prozessierungszeit korreliert ungefähr mit Unterschied der Sekundenzahl zwischen den Videos
			\item Anstieg der DCE Berechnungszeit lässt sich mit mehr Polygonen erkären, recht kurz trotzdem weil wenig vereinfacht werden muss (ähnliche Punktzahlen vor und nach DCE Durchlauf)
			\item Minderung bei SSM pro Frame und Boot hängt mit nicht so stark steigender absoluten SSM wie steigender Anzahl detektierte Boote zusammengesetzt
			\item steigende Anzahl detektierte Boote ist durch wachsende Länge zu erklären
		\end{itemize}
		\item geringe und hohe DCE Substitution
		\begin{itemize}
			\item Anzahl Punkte vor DCE ist gleich, weil Datensätze und YOLO Modell exakt gleich sind (YOLO ist deterministischer Algorithmus) (gilt auch für erkannte und vergl. Polygone und Absolute Anz. Auto/ LKW (bzw. auch pro Frame))
			\item Anzahl Punkte nach DCE ist so unterschiedlich, weil DCE auf verschieden große Punktgrenzen vereinfacht -> trivial weil am Ende wegen der niedrigeren/höheren Grenzwerte einfach weniger/mehr Punkte übrig bleiben
			\item betrifft auch die Zahl der vergl. Winkel bei SSM Berechnung (steigt exponentiell wg. Permutation)
			\item Gesamtwinkelsumme hängt an Punktzahl nach DCE -> daher so starker Anstieg
			\item Dauer der Prozessierung sinkt, weil DCE nicht mehr so viel vereinfachen muss
			\item Ähnliche Steigerungen bei SSM, weil  absolute SSM ja auch von Punktzahl abhängt, bzw. die unteren Werte ja von absoluter SSM
			\item Absolute Anzahl Auto/LKW bleibt gleich weil 
		\end{itemize}
		\item lange Testdatensätze
		\begin{itemize}
			\item 
		\end{itemize}
		\item Gleiche DCE Substitionslimits
		\begin{itemize}
			\item 
		\end{itemize}
	\end{itemize}



}
% {
% 	\begin{itemize}
		
% 		\item höhere SSMs bei als 'andere Objekte' deklarierte Polygone, da die Punktzahlen auf die diese reduziert werden höher ist, was zu einer höheren Gesamtwinkelsumme pro Polygon führt und damit die Abweichung zur nächsten Gesamtwinkelsumme im nächsten Frame steigt
% 		\item LKW werden aufgrund der langen Form mit recht monotonem Aussehen als Zug oder Bus Fehldetektiert
% 		\item LKW, die Autos transportieren können die absolute Anzahl der PKW stark verfälschen
% 		\item Falsche Detektierungen von Yolo werden durch Schilder und andere Objekte verursacht
% 	\end{itemize}

% }


