%!TEX root = ../thesis.tex
\chapter{Anhang}
\section{Weitere Abbildungen und Tabellen}
\subsection{Theoretischer Hintergrund}{

    % \begin{figure}[hb]
    % \centering
    % \includegraphics*[scale = 0.8, keepaspectratio, trim=2 2 2 2 ]{images/YOLO/YOLO_Preamble_PR_Curve.png}
    % \caption[Precision-/Recall- Kurve für eine Sequenz richtig-positver und falsch-positiver Detektionen]{Precision-/Recall- Kurve für eine Sequenz richtig-positver und falsch-positiver Detektionen. Diese Kurve ist unten abgebildet. Die falsch-positiven Detektionen sind für eine Objektklasse mit 6 Ground-Truth Instanzen absteigend oben geordnet. Die Sequenz der Precision und Recall Wertefelder ist als schwarze Kurve abgedruckt. Das Ergebnis des Ersetzen jeder Precision mit dem maximalen oder gleichen Recall ist im pinken Areal zu sehen. Die Average Precision ist das gesamte blaue und pinke Feld. Der Effekt der positiven Störung auf die Ergebnisse der falsch-positiven Erkennungen ist in den Pfeilen (a-e) abgebildet. Die blauen (a-c) Pfeile stellen die Störung ohne Effekt auf die Average Precision dar: (a) Es ist keine Auswirkung auf die Reihenfolge der Detektion zu erkennen. (b) Mit einer anderen falsch-positiven Detektion wird der Platz getauscht. (c) Der Platz der Detektion wird mit einer richtig-positiven Detektion, jedoch mit einem höheren Recall richtig positiven (f), der eine höhere Precision hat, getauscht, sodasss kein Wechsel des Feldes unter der pink schattierten Kurve erfolgt. Störungen, die einen Effekt auf die Average Precision haben sind orange Pfeile (d-e): (d) hat den gleichen falsch-positiven Wert wie (c) und wird über einen richtig-positiven Wert hinaus vershcoben, der auf der ausgefüllten Kurve erscheint (und somit Auswirkungen zeigt); (e) der falsch-positive Wert wird über einen einzelnen richtig-positiven Wert hinaus verschoben, woraufhin die ausgefüllte Kurve bis zu einem Abstand von 0,5 Recallpunkten verschiebt. \citep{Henderson2017}}
    % \label{YOLO_PR_Curve}
    % \end{figure}

    \begin{figure}[hb]
        \centering
        \includegraphics*[scale = 0.6, keepaspectratio, trim=2 2 2 2 ]{images/Loss_desc.png}
        \caption[Erklärung einer einfachen Loss Funktion]{Erklärung einer einfachen Loss Funktion: Im linken Modell ist der Loss hoch, im rechten Modell ist der Loss geringer. Verluste werden durch Pfeile dargestellt und die Vorhersage stellen die blauen Linien dar \citep{loss_google}. }
        \label{Loss_desc}
        \end{figure}
  
    \begin{figure}[h]
		\centering
		\includegraphics*[scale = 0.4, keepaspectratio]{images/YOLO/YOLO_loss_function_detail_expl.png}
		\caption[Detaillierte Erklärung der YOLO Loss Funktion]{Detaillierte Erklärung der YOLO Loss Funktion\citep{Terven2023}}
		\label{YOLO_loss_function_detail}
 	\end{figure}

    \begin{figure}[h]
		\centering
		\includegraphics*[scale = 0.15, keepaspectratio]{images/YOLO/YOLO_timeline_vers.png}
		\caption[Übersicht über verschiedene YOLO Versionen im Zeitverlauf]{Übersicht über verschiedene YOLO-Versionen im Zeitverlauf \citep{Terven2023}}
		\label{YOLO_timeline_vers}
		\end{figure}

    \begin{figure}[h]
        \centering
        \includegraphics*[scale = 0.35, keepaspectratio]{images/YOLO/YOLOv8_Arch.png}
        \caption[Die Architektur von YOLOv8]{Die Architektur von YOLOv8 \citep{Terven2023}}
        \label{YOLOv8_Arch}
    \end{figure}


    \begin{figure}[ht]
        \centering
        \includegraphics*[scale = 0.65, keepaspectratio, trim=2 2 2 2 ]{images/DCE/schem_maps_paper_DCE.png}
        \caption[Anwendungsbeispiele für die \glqq Discrete Curve Evolution\grqq{}]{Anwendungsbeispiele für die \glqq Discrete Curve Evolution\grqq{}  \citep{Barkowsky2000}.}
        \label{Bsp_DCE_Bark_Paper}
    \end{figure}


    \begin{figure}[ht]
		\centering
		\begin{subfigure}[b]{0.30\textwidth}
			\frame{\includegraphics[width=\textwidth]{images/DCE/kleines Beispiel erweitert/testpng0.png}}
            \caption{Polygon ohne Vereinfachung}
		\end{subfigure} 
		\begin{subfigure}[b]{0.30\textwidth}
            \frame{
			\includegraphics[width=\textwidth]{images/DCE/kleines Beispiel erweitert/testpng419.png}}
            \caption{Polygon (80 Punkte) }
		\end{subfigure}
        \begin{subfigure}[b]{0.30\textwidth}
			\frame{\includegraphics[width=\textwidth]{images/DCE/kleines Beispiel erweitert/testpng429.png}}
            \caption{Polygon (70 Punkte)}
		\end{subfigure}
        \begin{subfigure}[b]{0.30\textwidth}
			\frame{\includegraphics[width=\textwidth]{images/DCE/kleines Beispiel erweitert/testpng439.png}}
            \caption{Polygon (60 Punkte)}
		\end{subfigure}
        \begin{subfigure}[b]{0.30\textwidth}
			\frame{\includegraphics[width=\textwidth]{images/DCE/kleines Beispiel erweitert/testpng449.png}}
            \caption{Polygon (50 Punkte)}
		\end{subfigure}
        \begin{subfigure}[b]{0.30\textwidth}
			\frame{\includegraphics[width=\textwidth]{images/DCE/kleines Beispiel erweitert/testpng459.png}}
            \caption{Polygon (40 Punkte)}
		\end{subfigure}
        \begin{subfigure}[b]{0.30\textwidth}
			\frame{\includegraphics[width=\textwidth]{images/DCE/kleines Beispiel erweitert/testpng469.png}}
            \caption{Polygon (30 Punkte)}
		\end{subfigure}
        \begin{subfigure}[b]{0.30\textwidth}
			\frame{\includegraphics[width=\textwidth]{images/DCE/kleines Beispiel erweitert/testpng479.png}}
            \caption{Polygon (20 Punkte)}
		\end{subfigure}
        \begin{subfigure}[b]{0.30\textwidth}
			\frame{\includegraphics[width=\textwidth]{images/DCE/kleines Beispiel erweitert/testpng489.png}}
            \caption{Polygon (10 Punkte)}
		\end{subfigure}
		\caption[Anwendungsbeispiel für DCE aus eigenen Daten]{Anwendungsbeispiel für die DCE anhand eines Teilumrisses von Nordrhein Westfalen; (a) zeigt das Ursprungspolygon mit 499 Punkten; (b) bis (i) zeigen um je 10 Punkte reduzierte Polygone, von 80 Punkten bis 10 Punkten (Quelle: eigene Darstellung)}
		\label{Scr:DCE_test_run_nrw}
	\end{figure}





    \clearpage
\subsection{Evaluation}{


\begin{table}[h]
    \centering
\caption[Vergleich der verschiedenen YOLO Modelle bei 1 Sekunde Video (30 Frames)]{Vergleich der verschiedenen YOLO Modelle bei 1 Sekunde Video (30 Frames) (Quelle: eigene Darstellung; \ref{cd:listing_A1s_EF_results.txt(Y8n)}; \ref{cd:listing_A1s_RV_results.txt(Y8n)}; \ref{cd:listing_A1s_EF_results.txt(Y8m)}; \ref{cd:listing_A1s_RV_results.txt(Y8m)}; \ref{cd:listing_A1s_EF_results.txt(Y8x)}; \ref{cd:listing_A1s_RV_results.txt(Y8x)})}
\label{tab:YOLO8_A1s}
\begin{tabular}{l|l|l|l|l|l|l}
    & \textbf{8n (EFV)} & \textbf{8n (RV)} & \textbf{8m (EFV)} & \textbf{8m (RV)} & \textbf{8x (EFV)} & \textbf{8x (RV)} \\ \hline
\textit{\begin{tabular}[c]{@{}l@{}}Anzahl Punkte/Winkel\\ vor DCE\end{tabular}} & 5.353 & 5.303 & 5.863 & 5.885 & 5.306 & 5.325 \\ \hline
\textit{\begin{tabular}[c]{@{}l@{}}Anzahl Punkte/Winkel\\ nach DCE\end{tabular}} & 1.919 & 1.880 & 2.168 & 2.185 & 1.965 & 1.988 \\ \hline
\textit{\begin{tabular}[c]{@{}l@{}}Anzahl vergl. Winkel\\ bei SSM Berechnung\end{tabular}} & 19.100 & 17.811 & 10.251 & 10.130 & 11.803 & 11.858 \\ \hline
\textit{} &  &  &  &  &  &  \\ \hline
\textit{\begin{tabular}[c]{@{}l@{}}Gesamtwinkelsumme\\ (in Deg.)\end{tabular}} & 6.020,32 & 5.898,2 & 6.802,79 & 6.856,27 & 6.166,18 & 6.238,21 \\ \hline
\textit{erkannte Polygone} & 147 & 145 & 209 & 210 & 212 & 214 \\ \hline
\textit{verglichene Polygone} & 131 & 129 & 188 & 189 & 193 & 194 \\ \hline
\textit{Prozessierungszeit (in Sek.)} & 29,54 & 20,19 & 47,92 & 30.64 & 80,15 & 44,09
\end{tabular}
\end{table}


    \begin{table}[ht]
		\caption[Vergleich der SSMs bei verschiedenen YOLO Modellen bei 1 Sekunde Video (30 Frames)]{Vergleich der SSMs bei verschiedenen YOLO Modellen bei 1 Sekunde Video (30 Frames) (Quelle: eigene Darstellung; \ref{cd:listing_A10s_EF_results.txt(Y8n)}; \ref{cd:listing_A10s_RV_results.txt(Y8n)}; \ref{cd:listing_A10s_EF_results.txt(Y8m)}; \ref{cd:listing_A10s_RV_results.txt(Y8m)}; \ref{cd:listing_A10s_EF_results.txt(Y8x)}; \ref{cd:listing_A10s_RV_results.txt(Y8x)})}
		\label{tab:YOLO8_A1s_SSM}
		\begin{tabular}{l|l|l|l|l|l|l}
		 & \textbf{8n (EFV)} & \textbf{8n (RV)} & \textbf{8m (EFV)} & \textbf{8m (RV)} & \textbf{8x (EFV)} & \textbf{8x (RV)} \\ \hline
		\textit{Absolute SSM Auto} & 2,0174 & 2,0174 & 2,2393 & 2,3659 & 3,8058 & 4,1018 \\ \hline
		\textit{SSM pro Fr. und Kl. Auto} & 0,0672 & 0,0672 & 0,0746 & 0,0789 & 0,1269 & 0,1637 \\ \hline
		\textit{SSM pro detektiertes Auto} & 0,0672 & 0,0721 & 0,0622 & 0,0676 & 0,0634 & 0,0672 \\ \hline
		\textit{\begin{tabular}[c]{@{}l@{}}absolute Anz. detektierter\\ Autos (in Klam. pro Fr.)\end{tabular}} & \begin{tabular}[c]{@{}l@{}}30\\ (1,00)\end{tabular} & \begin{tabular}[c]{@{}l@{}}28\\ (0,93)\end{tabular} & \begin{tabular}[c]{@{}l@{}}36\\ (1,2)\end{tabular} & \begin{tabular}[c]{@{}l@{}}35\\ (1,17)\end{tabular} & \begin{tabular}[c]{@{}l@{}}60\\ (2,00)\end{tabular} & \begin{tabular}[c]{@{}l@{}}61\\ (2,23)\end{tabular} \\ \hline
		 &  &  &  &  &  &  \\ \hline
		\textit{Absolute SSM LKW} & 4,4640 & 4,6617 & 13,3996 & 14,2144 & 11,5296 & 11,8367 \\ \hline
		\textit{SSM pro Fr. und Kl. LKW} & 0,1488 & 0,1554 & 0,4467 & 0,4738 & 0,3843 & 0,3946 \\ \hline
		\textit{SSM pro detektierter LKW} & 0,2349 & 0,2454 & 0,2197 & 0,2330 & 0,1696 & 0,1767 \\ \hline
		\textit{\begin{tabular}[c]{@{}l@{}}absolute Anz. detektierter\\ LKW (in Klam. pro Fr.)\end{tabular}} & \begin{tabular}[c]{@{}l@{}}19\\ (0,63)\end{tabular} & \begin{tabular}[c]{@{}l@{}}19\\ (0,63)\end{tabular} & \begin{tabular}[c]{@{}l@{}}61\\ (2,03)\end{tabular} & \begin{tabular}[c]{@{}l@{}}61\\ (2,03)\end{tabular} & \begin{tabular}[c]{@{}l@{}}68\\ (2,27)\end{tabular} & \begin{tabular}[c]{@{}l@{}}67\\ (2,23)\end{tabular} \\ \hline
		 &  &  &  &  &  &  \\ \hline
		\textit{Absolute SSM Zug} & 11,8231 & 11,8231 &  &  &  &  \\ \hline
		\textit{SSM pro Fr. und Kl. Zug} & 0,3941 & 0,3941 &  &  &  &  \\ \hline
		\textit{SSM pro detektierten Zug} & 0,3111 & 0,3378 &  &  &  &  \\ \hline
		\textit{\begin{tabular}[c]{@{}l@{}}absolute Anz. detektierter\\ Züge (in Klam. pro Fr.)\end{tabular}} & \begin{tabular}[c]{@{}l@{}}38\\ (1,27)\end{tabular} & \begin{tabular}[c]{@{}l@{}}35\\ (1,17)\end{tabular} &  &  &  &  \\ \hline
		 &  &  &  &  &  &  \\ \hline
		\textit{Absolute SSM Bus} &  &  & 1,8576 & 1,8576 &  &  \\ \hline
		\textit{SSM pro Fr. und Kl. Bus} &  &  & 0,0619 & 0,0619 &  &  \\ \hline
		\textit{SSM pro detektiertem Bus} &  &  & 0,4644 & 0,4644 &  &  \\ \hline
		\textit{\begin{tabular}[c]{@{}l@{}}absolute Anz. detektierter\\ Busse (in Klam. pro Fr.)\end{tabular}} &  &  & \begin{tabular}[c]{@{}l@{}}4\\ (0,13)\end{tabular} & \begin{tabular}[c]{@{}l@{}}4\\ (0,13)\end{tabular} &  & 
		\end{tabular}
		\end{table}
        \clearpage

        \subsubsection{Geringe und hohe DCE Substitionslimits}
            \begin{table}[ht]
                \centering
                \caption[Vergleich der Basisdaten bei 1 Sekunde Video (30 Frames) und verschiedenen DCE Substitutionsgrenzen]{Vergleich der Basisdaten bei 1 Sekunde Video (30 Frames) und verschiedenen DCE Substitutionsgrenzen (Quelle: eigene Darstellung; \ref{cd:listing_A1s_RV_minor_results.txt(Y8x)}, \ref{cd:listing_A1s_RV_more_results.txt(Y8x)})}
                \label{tab:YOLO8_minor_more_DCE_A1s}
                \begin{tabular}{l|l|l|l}
                    & \textbf{\begin{tabular}[c]{@{}l@{}}geringe DCE\\ Punktgrenzen\end{tabular}} & \textbf{Referenz} & \textbf{\begin{tabular}[c]{@{}l@{}}hohe DCE\\ Punktgrenzen\end{tabular}} \\ \hline
                   \textit{\begin{tabular}[c]{@{}l@{}}Anzahl Punkte/Winkel \\ vor DCE\end{tabular}} & 5.325 & 5.325 & 5.325 \\ \hline
                   \textit{\begin{tabular}[c]{@{}l@{}}Anzahl Punkte/Winkel\\ nach DCE\end{tabular}} & 1.174 & 1.988 & 4.026 \\ \hline
                   \textit{\begin{tabular}[c]{@{}l@{}}Anzahl vergl. Winkel\\ bei SSM Berechnung\end{tabular}} & 3.987 & 11.858 & 74.312 \\ \hline
                   \textit{} &  &  &  \\ \hline
                   \textit{Gesamtwinkelsumme} & 3.688,93 & 6.238,21 & 12.630,05 \\ \hline
                   \textit{erkannte Polygone} & 214 & 214 & 214 \\ \hline
                   \textit{verglichene Polygone} & 194 & 194 & 194 \\ \hline
                   \textit{Prozessierungszeit (in Sek.)} & 45,24 & 44,09 & 43,45
                   \end{tabular}
                \end{table}
    
                \begin{table}[ht]
                    \caption[Vergleich der SSMs bei verschiedenen DCE Substitionslimits bei einem 1-sekündigem Video (30 Frames) ]{Vergleich der SSMs bei verschiedenen DCE Substitionslimits bei einem 1-sekündigem Video (30 Frames) (Quelle: eigene Darstellung; \ref{cd:listing_A1s_RV_minor_results.txt(Y8x)}, \ref{cd:listing_A1s_RV_more_results.txt(Y8x)})}
                    \label{tab:Minor_More_DCE_SSMs_A1s}
                    \centering
                    \begin{tabular}{l|l|l|l}
                        & \textbf{\begin{tabular}[c]{@{}l@{}}geringe DCE\\ Punktgrenzen\end{tabular}} & \textbf{Referenz} & \textbf{\begin{tabular}[c]{@{}l@{}}hohe DCE\\ Punktgrenzen\end{tabular}} \\ \hline
                       \textit{Absolute SSM Auto} & 3,0831 & 4,1018 & 7,7453 \\ \hline
                       \textit{SSM pro Fr. und Kl. Auto} & 0,1028 & 0,1367 & 0,2582 \\ \hline
                       \textit{SSM pro detektiertes Auto} & 0,0505 & 0,0672 & 0,1270 \\ \hline
                       \textit{\begin{tabular}[c]{@{}l@{}}absolute Anz. detektierter\\ Autos (in Kl. pr. Fr.)\end{tabular}} & \begin{tabular}[c]{@{}l@{}}61\\ (2,03)\end{tabular} & \begin{tabular}[c]{@{}l@{}}61\\ (2,03)\end{tabular} & \begin{tabular}[c]{@{}l@{}}61\\ (2,03)\end{tabular} \\ \hline
                       \textit{} &  &  &  \\ \hline
                       \textit{Absolute SSM LKW} & 9,4661 & 11,8367 & 22,6788 \\ \hline
                       \textit{SSM pro Fr. und Kl. LKW} & 0,3155 & 0,3946 & 0,7560 \\ \hline
                       \textit{SSM pro detektierten LKW} & 0,1413 & 0,1767 & 0,3385 \\ \hline
                       \textit{\begin{tabular}[c]{@{}l@{}}Absolute Anz. detektierter\\ LKW (in Kl. pro Fr.)\end{tabular}} & \begin{tabular}[c]{@{}l@{}}67\\ (2,23)\end{tabular} & \begin{tabular}[c]{@{}l@{}}67\\ (2,23)\end{tabular} & \begin{tabular}[c]{@{}l@{}}67\\ (2,23)\end{tabular}
                       \end{tabular}
                    \end{table}	
                    
                
            
}
}
\clearpage
\lstset{commentstyle=\color{black}, stepnumber=2, basicstyle=\ttfamily\scriptsize} 
\section{Listing der Evaluationsergebnisse}
\subsection{Listing der Result Textdaten für YOLO8n Testdurchläufe}{ \label{ls:result_txts}
    \lstinputlisting[caption={A1s\_EF\_results.txt (YoloV8n)}, label = {cd:listing_A1s_EF_results.txt(Y8n)}, language={}]{../Code/vid_examples/evaluation/01Yolo8n/A1s_EF_results.txt}
    \lstinputlisting[caption={A1s\_RV\_results.txt (YoloV8n)}, label = {cd:listing_A1s_RV_results.txt(Y8n)}, language={}]{../Code/vid_examples/evaluation/01Yolo8n/A1s_RV_results.txt}
    \lstinputlisting[caption={A10s\_EF\_results.txt (YoloV8n)}, label = {cd:listing_A10s_EF_results.txt(Y8n)}, language={}]{../Code/vid_examples/evaluation/01Yolo8n/A10s_EF_results.txt}
    \lstinputlisting[caption={A10s\_RV\_results.txt (YoloV8n)}, label = {cd:listing_A10s_RV_results.txt(Y8n)}, language={}]{../Code/vid_examples/evaluation/01Yolo8n/A10s_RV_results.txt}
}
\subsection{Listing der Result Textdaten für YOLO8m Testdurchläufe}{
    \lstinputlisting[caption={A1s\_EF\_results.txt (YoloV8m)}, label = {cd:listing_A1s_EF_results.txt(Y8m)}, language={}]{../Code/vid_examples/evaluation/02Yolo8m/A1s_EF_results.txt}
    \lstinputlisting[caption={A1s\_RV\_results.txt (YoloV8m)}, label = {cd:listing_A1s_RV_results.txt(Y8m)}, language={}]{../Code/vid_examples/evaluation/02Yolo8m/A1s_RV_results.txt}
    \lstinputlisting[caption={A10s\_EF\_results.txt (YoloV8m)}, label = {cd:listing_A10s_EF_results.txt(Y8m)}, language={}]{../Code/vid_examples/evaluation/02Yolo8m/A10s_EF_results.txt}
    \lstinputlisting[caption={A10s\_RV\_results.txt (YoloV8m)}, label = {cd:listing_A10s_RV_results.txt(Y8m)}, language={}]{../Code/vid_examples/evaluation/02Yolo8m/A10s_RV_results.txt}
}
\subsection{Listing der Result Textdaten für YOLO8x Testdurchläufe}{
    \lstinputlisting[caption={A1s\_EF\_results.txt (YoloV8x)}, label = {cd:listing_A1s_EF_results.txt(Y8x)}, language={}]{../Code/vid_examples/evaluation/03Yolo8x/A1s_EF_results.txt}
    \lstinputlisting[caption={A1s\_RV\_results.txt (YoloV8x)}, label = {cd:listing_A1s_RV_results.txt(Y8x)}, language={}]{../Code/vid_examples/evaluation/03Yolo8x/A1s_RV_results.txt}
    \lstinputlisting[caption={A10s\_EF\_results.txt (YoloV8x)}, label = {cd:listing_A10s_EF_results.txt(Y8x)}, language={}]{../Code/vid_examples/evaluation/03Yolo8x/A10s_EF_results.txt}
    \lstinputlisting[caption={A10s\_RV\_results.txt (YoloV8x)}, label = {cd:listing_A10s_RV_results.txt(Y8x)}, language={}]{../Code/vid_examples/evaluation/03Yolo8x/A10s_RV_results.txt}
}
\subsection{Listing der Result Textdaten der weiteren Testfälle}{
    \subsubsection{Schiffstracking}{
        \lstinputlisting[caption={S1s\_RV\_10.txt (YoloV8x, Schiffstracking)}, label = {cd:listing_A1s_RV_shiptracking_results.txt(Y8x)}, language={}]{../Code/vid_examples/evaluation/weitere_testfaelle/shiptracking/A1s_RV_10.txt}
        \lstinputlisting[caption={S21s\_RV\_10.txt (YoloV8x, Schiffstracking)}, label = {cd:listing_A21s_RV_shiptracking_results.txt(Y8x)}, language={}]{../Code/vid_examples/evaluation/weitere_testfaelle/shiptracking/A21s_RV_10.txt}
    }
    \subsubsection{Geringe und hohe DCE Substitionslimits}{
        \lstinputlisting[caption={A1s\_RV\_minor.txt (YoloV8x, geringes DCE Substitionslimit)}, label = {cd:listing_A1s_RV_minor_results.txt(Y8x)}, language={}]{../Code/vid_examples/evaluation/weitere_testfaelle/minor_more_DCEPoints/A1s_RV_minor.txt}  
        \lstinputlisting[caption={A1s\_RV\_more.txt (YoloV8x, hohes DCE Substitionslimit)}, label = {cd:listing_A1s_RV_more_results.txt(Y8x)}, language={}]{../Code/vid_examples/evaluation/weitere_testfaelle/minor_more_DCEPoints/A1s_RV_more.txt}     
        \lstinputlisting[caption={A10s\_RV\_minor.txt (YoloV8x, geringes DCE Substitionslimit)}, label = {cd:listing_A10s_RV_minor_results.txt(Y8x)}, language={}]{../Code/vid_examples/evaluation/weitere_testfaelle/minor_more_DCEPoints/A10s_RV_minor.txt}  
        \lstinputlisting[caption={A10s\_RV\_more.txt (YoloV8x, hohes DCE Substitionslimit)}, label = {cd:listing_A10s_RV_more_results.txt(Y8x)}, language={}]{../Code/vid_examples/evaluation/weitere_testfaelle/minor_more_DCEPoints/A10s_RV_more.txt} 

    }
    \subsubsection{Langer Testdatensatz}{
        \lstinputlisting[caption={A30s\_RV\_results.txt (YoloV8x, 30-sekündiger Testdatensatz)}, label = {cd:listing_A30s_RV_results.txt(Y8x)}, language={}]{../Code/vid_examples/evaluation/weitere_testfaelle/longshots/A30s_RV_results.txt}  

    }
    \subsubsection{Gleiche DCE Substitionslimits}{
        \lstinputlisting[caption={A10s\_RV\_minorSame\_results.txt (YoloV8x, gleich wenige DCE Substitionslimits)}, label = {cd:listing_A10s_RV_minorSame_results.txt(Y8x)}, language={}]{../Code/vid_examples/evaluation/weitere_testfaelle/sameDCESubst/A10s_RV_minorSame_results.txt} 
        \lstinputlisting[caption={A10s\_RV\_moreSame\_results.txt (YoloV8x, gleich hohe DCE Substitionslimits)}, label = {cd:listing_A10s_RV_moreSame_results.txt(Y8x)}, language={}]{../Code/vid_examples/evaluation/weitere_testfaelle/sameDCESubst/A10s_RV_moreSame_results.txt} 


    }

}
\section{Listing mit Dokumentation und Kommentaren}{\label{cd:gesamt_listing}}


\todo{More Comment auf Default setzen noch einbauen, oder doch oben kommentare nur weiß setzten nach lattenzaun; vielleicht kurzbeschreibungen noch klarer machen (zu viele 'Auschnitt aus...' Bennenungen)}
\todo{YOLO Segmentation. py auch?}


\subsection{Vollständiges Listing von Main.py}{
    \lstinputlisting[caption={Main.py}, label = {cd:listing_main.py}]{../Code/main.py}}

\subsection{Vollständiges Listing von yolo\_every\_frame.py}{
    \lstinputlisting[caption={YOLO\_every\_frame.py}, label = {cd:listing_yolo_every_frame_py}]{../Code/YOLO/yolo_every_frame.py}}

\subsection{Vollständiges Listing von YOLO\_result\_version.py}{
    \lstinputlisting[caption={YOLO\_result\_version.py}, label = {cd:listing_yolo_result_version}]{../Code/YOLO/yolo_result_version.py}}

\subsection{Vollständiges Listing von YOLO\_segmentation.py}{
    \lstinputlisting[caption={YOLO\_segementation.py}, label = {cd:listing_yolo_segmentation.py}]{../Code/YOLO/yolo_segmentation.py}}

\subsection{Vollständiges Listing von DCE.py}{
    \lstinputlisting[caption={DCE.py}, label = {cd:listing_DCE.py}]{../Code/DCE/DCE.py}}

\subsection{ Vollständiges Listing von Shape\_Similarity\_meas.py}{
    \lstinputlisting[caption={shape\_sim\_meas.py}, label = {cd:listing_shape_sim_meas.py}]{../Code/Shape_Similiarity/shape_sim_meas.py}
}

