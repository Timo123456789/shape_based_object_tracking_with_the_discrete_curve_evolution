%!TEX root = ../thesis.tex
\chapter{Evaluation}
\label{ch:Evaluation}

{
	\begin{itemize}
		\item Evaluation mit verschiedenen Testvideos
		\item immer das gleiche Video nur mit verschiedenen Längen
		\item Dann verschiedene Berechnungsmethode (YOLO every frame, yolo result)
		\item mit und ohne schwarzes video ?
		\item mit verschiedenen YOLO Modellen (v8n, v8x, v8)
	\end{itemize}

}
\section{Testumgebung}{
	Laptop mit x specs
	Laptop ist an Strom (65 Watt Netzteil angeschlossen), weitere Specs
}
\section{Testfälle}
{Vergleich von Dauer etc. anderen timestamp variablen, vielleicht auch andere Endpunktzahlen bei den Polygonen
\begin{table}[]
	\caption{Vergleich der verschiedenen YOLO8 bei 1 Sekunde Video (30 Frames)}
	\label{tab:YOLO8_A1s}
	\begin{tabular}{l|l|l|l|l|l|l}
	 & \textbf{8n (EFV)} & \textbf{8n (RV)} & \textbf{8m (EFV)} & \textbf{8m (RV)} & \textbf{8x (EFV)} & \textbf{8x (RV)} \\ \hline
	\textit{\begin{tabular}[c]{@{}l@{}}Durchschn. Abweichung\\ p. Polygon (in Deg.)\end{tabular}} & 26,49 & 26,97 & 35,19 & 35,04 & 44,33 & 48,12 \\ \hline
	\textit{\begin{tabular}[c]{@{}l@{}}Durchschn. Abweichung  \\ p. Winkel (in Deg.)\end{tabular}} & 2,04 & 2,09 & 3,59 & 3,56 & 5,05 & 5,46 \\ \hline
	\textit{} &  &  &  &  &  &  \\ \hline
	\textit{\begin{tabular}[c]{@{}l@{}}absolute Abweichung  \\ (in Deg)\end{tabular}} & 3894,53 & 3910,4 & 7354,98 & 7358,55 & 9398,12 & 10298,28 \\ \hline
	\textit{erkannte Punkte/Winkel} & 1912 & 1873 & 2050 & 2065 & 1860 & 1885 \\ \hline
	\textit{erkannte Polygone} & 147 & 145 & 209 & 210 & 212 & 214 \\ \hline
	\textit{verglichene Polygone} & 131 & 129 & 188 & 189 & 193 & 194 \\ \hline
	\textit{Prozessierungszeit (in Sek.)} & 92,52 & 81,51 & 111,6 & 100,03 & 134,9 & 102,46
	\end{tabular}
	\end{table}
	}

	\begin{table}[]
		\caption{Vergleich der verschiedenen YOLO8 bei 10 Sekunde Video (300 Frames)}
		\label{tab:YOLO8_A10s}
		\begin{tabular}{l|l|l|l|l|l|l}
		 & \textbf{8n (EFV)} & \textbf{8n (RV)} & \textbf{8m (EFV)} & \textbf{8m (RV)} & \textbf{8x (EFV)} & \textbf{8x (RV)} \\ \hline
		\textit{\begin{tabular}[c]{@{}l@{}}Durchschn. Abw.\\ p. Polygon (in Deg.)\end{tabular}} & 53,07 & 53,11 & 49,76 & 49,79 & 57,93 & 58,04 \\ \hline
		\textit{\begin{tabular}[c]{@{}l@{}}Durchschn. Abw.  \\ p. Winkel (in Deg.)\end{tabular}} & 4,29 & 4,29 & 5,21 & 5,21 & 6,59 & 6,6 \\ \hline
		\textit{} &  &  &  &  &  &  \\ \hline
		\textit{\begin{tabular}[c]{@{}l@{}}absolute Abw. \\ (in Deg)\end{tabular}} & 73.075,23 & 73.075,23 & 100.561,26 & 100.579,53 & 122.111,04 & 122.133,17 \\ \hline
		\textit{erk. Punkte/Winkel} & 17.043 & 17.042 & 19.298 & 19.298 & 18.534 & 18.498 \\ \hline
		\textit{erk.  Polygone} & 1.377 & 1.376 & 2.021 & 2.020 & 2.108 & 2.104 \\ \hline
		\textit{verglichene Polygone} & 1.235 & 1.234 & 1.871 & 1.870 & 1.964 & 1960 \\ \hline
		\textit{\begin{tabular}[c]{@{}l@{}}Prozessierungszeit\\ (in Min.)\end{tabular}} & 23,56 & 22,16 & 26,39 & 22.25 & 28,48 & 23,84
		\end{tabular}
		\end{table}


		\begin{table}[]
			\caption{Vergleich der SSMs bei verschiedenen YOLO8 bei 1 Sekunde Video (30 Frames)}
			\label{tab:YOLO8_A10s}
			\begin{tabular}{ll|l|l|l|l|l}
			\multicolumn{1}{l|}{} & \textbf{8n (EFV)} & \textbf{8n (RV)} & \textbf{8m (EFV)} & \textbf{8m (RV)} & \textbf{8x (EFV)} & \textbf{8x (RV)} \\ \hline
			\multicolumn{1}{l|}{\textit{SSM pro Fr. und Kl. Auto}} & 55,67 & 55,69 & 258,55 & 270,58 & 343,47 & 355,41 \\ \hline
			\multicolumn{1}{l|}{\textit{SSM pro detektiertes Auto}} & 30,93 & 30,94 & 95,76 & 100,22 & 101,02 & 102,52 \\ \hline
			\multicolumn{1}{l|}{\textit{\begin{tabular}[c]{@{}l@{}}absolute Anz. detektierter\\ Autos (in Klam. pro Fr.)\end{tabular}}} & 54 (1,8) & 54 (1,8)) & 81 (2,7) & 81 (2,7) & 102 (3,4) & 104 (3,47) \\ \hline
			\multicolumn{1}{l|}{} &  &  &  &  &  &  \\ \hline
			\textit{SSM pro Fr. und Kl. LKW} & 1,40 & 1,40 & 15,66 & 15,38 & 14,32 & 14,19 \\ \hline
			\multicolumn{1}{l|}{\textit{SSM pro detektierter LKW}} & 1,14 & 1,14 & 4,75 & 4,66 & 4,17 & 4,09 \\ \hline
			\multicolumn{1}{l|}{\textit{\begin{tabular}[c]{@{}l@{}}absolute Anz. detektierter\\ LKW (in Klam. pro Fr.)\end{tabular}}} & 37 (1,23) & 37 (1,23) & 99 (3,3) & 99 (3,3) & 103 (3,43) & 104 (3,47) \\ \hline
			\multicolumn{1}{l|}{} &  &  &  &  &  &  \\ \hline
			\multicolumn{1}{l|}{\textit{SSM pro Fr. und Kl. Zug}} & 125,37 & 119,29 & - & - & - & - \\ \hline
			\multicolumn{1}{l|}{\textit{SSM pro detektierten Zug}} & 70,96 & 70,17 & - & - & - & - \\ \hline
			\multicolumn{1}{l|}{\textit{\begin{tabular}[c]{@{}l@{}}absolute Anz. detektierter\\ Züge (in Klam. pro Fr.)\end{tabular}}} & 53 (1,77) & 51 (1,7) & - & - & - & - \\ \hline
			\multicolumn{1}{l|}{} &  &  &  &  &  &  \\ \hline
			\multicolumn{1}{l|}{SSM pro Fr. und Kl. Bus} & - & - & 71,83 & 71,83 & - & - \\ \hline
			\multicolumn{1}{l|}{SSM pro detektiertem Bus} & - & - & 126,77 & 126,77 & - & - \\ \hline
			\multicolumn{1}{l|}{\begin{tabular}[c]{@{}l@{}}absolute Anz. detektierter\\ Busse (in Klam. pro Fr.)\end{tabular}} & - & - & 17 (0,57) & 17 (0,57) & - & -
			\end{tabular}
			\end{table}

			\begin{table}[]
				\caption{Vergleich der SSMs bei verschiedenen YOLO8 bei 10 Sekunde Video (300 Frames)}
				\label{tab:YOLO8_A10s}
				\begin{tabular}{l|l|l|l|l|l|l}
				 & \textbf{8n (EFV)} & \textbf{8n (RV)} & \textbf{8m (EFV)} & \textbf{8m (RV)} & \textbf{8x (EFV)} & \textbf{8x (RV)} \\ \hline
				\textit{SSM pro Fr. und Kl. Auto} & 34,25 & 34,24 & 232,00 & 232,00 & 344,68 & 344,68 \\ \hline
				\textit{SSM pro detektiertes Auto} & 21,31 & 21,36 & 87,00 & 87,22 & 114,13 & 114,38 \\ \hline
				\textit{\begin{tabular}[c]{@{}l@{}}absolute Anz. detektierter\\ Autos (in Klam. pro Fr.)\end{tabular}} & \begin{tabular}[c]{@{}l@{}}482 \\ (1,61)\end{tabular} & \begin{tabular}[c]{@{}l@{}}481 \\ (1,6)\end{tabular} & \begin{tabular}[c]{@{}l@{}}800\\ (2,67)\end{tabular} & \begin{tabular}[c]{@{}l@{}}798 \\ (2,66)\end{tabular} & \begin{tabular}[c]{@{}l@{}}906\\ (3,02)\end{tabular} & \begin{tabular}[c]{@{}l@{}}904 \\ (3,63)\end{tabular} \\ \hline
				 &  &  &  &  &  &  \\ \hline
				\textit{SSM pro Fr. und Kl. LKW} & 1,90 & 1,90 & 9,01 & 9,00 & 12,86 & 12,86 \\ \hline
				\textit{SSM pro detektierter LKW} & 1,47 & 1,47 & 2,96 & 2,95 & 3,55 & 3,54 \\ \hline
				\textit{\begin{tabular}[c]{@{}l@{}}absolute Anz. detektierter\\ LKW (in Klam. pro Fr.)\end{tabular}} & \begin{tabular}[c]{@{}l@{}}388 \\ (1,29)\end{tabular} & \begin{tabular}[c]{@{}l@{}}387\\ (1,29)\end{tabular} & \begin{tabular}[c]{@{}l@{}}914 \\ (3,05)\end{tabular} & \begin{tabular}[c]{@{}l@{}}914\\ (3,05)\end{tabular} & \begin{tabular}[c]{@{}l@{}}1088\\ (3,63)\end{tabular} & \begin{tabular}[c]{@{}l@{}}1088 \\ (3,63)\end{tabular} \\ \hline
				 &  &  &  &  &  &  \\ \hline
				\textit{SSM pro Fr. und Kl. Zug} & 14,86 & 14,28 & - & - & F & F \\ \hline
				\textit{SSM pro detektierten Zug} & 20,44 & 19,65 & - & - & F & F \\ \hline
				\textit{\begin{tabular}[c]{@{}l@{}}absolute Anz. detektierter\\ Züge (in Klam. pro Fr.)\end{tabular}} & \begin{tabular}[c]{@{}l@{}}218 \\ (0,73)\end{tabular} & \begin{tabular}[c]{@{}l@{}}218\\ (0,73)\end{tabular} & - & - & \begin{tabular}[c]{@{}l@{}}7\\ (0,02)\end{tabular} & \begin{tabular}[c]{@{}l@{}}7\\ (0,02)\end{tabular} \\ \hline
				 &  &  &  &  &  &  \\ \hline
				\textit{SSM pro Fr. und Kl. Bus} & 0,0004 & 0,0004 & 8,41 & 8,41 & F & F \\ \hline
				\textit{SSM pro detektiertem Bus} & 0,0023 & 0,0023 & 37,10 & 36,56 & F & F \\ \hline
				\textit{\begin{tabular}[c]{@{}l@{}}absolute Anz. detektierter\\ Busse (in Klam. pro Fr.)\end{tabular}} & \begin{tabular}[c]{@{}l@{}}58 \\ (0,19)\end{tabular} & \begin{tabular}[c]{@{}l@{}}58\\ (0,19)\end{tabular} & \begin{tabular}[c]{@{}l@{}}68 \\ (0,23)\end{tabular} & \begin{tabular}[c]{@{}l@{}}69 \\ (0,23)\end{tabular} & \begin{tabular}[c]{@{}l@{}}7\\ (0,02)\end{tabular} & \begin{tabular}[c]{@{}l@{}}7\\ (0,02)\end{tabular} \\ \hline
				\textit{} &  &  &  &  &  &  \\ \hline
				\textit{\begin{tabular}[c]{@{}l@{}}SSM pro Fr. und Kl. \\ Motorrad\end{tabular}} & - & - & 0,02 & 0,02 & - & - \\ \hline
				\textit{\begin{tabular}[c]{@{}l@{}}SSM pro detektiertes \\ Motorrad\end{tabular}} & - & - & 1,30 & 1,30 & - & - \\ \hline
				\textit{\begin{tabular}[c]{@{}l@{}}absolute Anz. detektierter\\ Motorräder (in Kl. pro Fr.)\end{tabular}} & - & - & \begin{tabular}[c]{@{}l@{}}4\\ (0,01)\end{tabular} & \begin{tabular}[c]{@{}l@{}}4 \\ (0,01)\end{tabular} & - & -
				\end{tabular}
				\end{table}
\subsection{yolo8n}
{black video immer an, yolo8n}
\

{black video immer an, yolo8m}


\subsection{yolo8x}
{black video immer an, yolo8x}




\section{Bewertung der Testfälle im Vergleich}




\section{Ausblick}
\begin{enumerate}
	\item DCE Abbruchbedingung nicht fest implementieren sondern einen Wert einführen, der immer im Vergleich zur Ähnlichkeit des Ursprungspolygons gemessen wird. Ermöglicht eine bedarfsbezogene Vereinfachung des Polygons, individuell für jeden erkannten Umriss
	\item Implementierung der DCE; bzw. anderer Programmteile in schnellerer Programmiersprache wie C/C++; bzw. hardwarenah(er)
\end{enumerate}
{
	Wenn das Ergebnis der Anwendung von DCE für Objekttracking erfolgreich ist und die getrackten Objekte erkennbar bleiben, kann eine hardwarenahe Programmierung erfolgen. Diese könnte in C oder C++ gemacht werden, um schnellere Ergebnisse liefern zu können, da die Prozessierungsgeschwindigkeit von Python begrenzt ist.\newline
Durch die hardwarenahe Implementierung der DCE könnte eine Anonymisierung direkt am Aufzeichnungsort stattfinden. Auch ein vereinfachtes Objekttracking ist durch eine erfolgreiche Anwendung der DCE möglich.


}



