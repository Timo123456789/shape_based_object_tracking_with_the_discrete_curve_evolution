%!TEX root = ../thesis.tex
\chapter{Evaluation}
\label{ch:Evaluation}

{
	\begin{itemize}
		\item Evaluation mit verschiedenen Testvideos
		\item immer das gleiche Video nur mit verschiedenen Längen
		\item Dann verschiedene Berechnungsmethode (YOLO every frame, yolo result)
		\item mit und ohne schwarzes video ?
		\item mit verschiedenen YOLO Modellen (v8n, v8x, v8)
	\end{itemize}

}
\section{Testumgebung}{
	Laptop mit x specs
	Laptop ist an Strom (65 Watt Netzteil angeschlossen), weitere Specs
}
\section{Testfälle}
{Vergleich von Dauer etc. anderen timestamp variablen, vielleicht auch andere Endpunktzahlen bei den Polygonen
\begin{table}[]
	\caption{Vergleich der verschiedenen YOLO Modelle bei 1 Sekunde Video (30 Frames)}
	\label{tab:YOLO_Modell_A1s}
	\begin{tabular}{|l|l|l|l|l|l|l|}
	\hline
	\textit{} & {\ul \textit{\textbf{8n (EFV)}}} & {\ul \textit{\textbf{8n (RV)}}} & {\ul \textit{\textbf{8m (EFV)}}} & {\ul \textit{\textbf{8m (RV)}}} & {\ul \textit{\textbf{8x (EFV)}}} & {\ul \textit{\textbf{8x(RV)}}} \\ \hline
	\textit{\begin{tabular}[c]{@{}l@{}}Durchschn. Abweichung\\ p. Polygon (in Deg.)\end{tabular}} & 26,49 & 26,97 & 35,19 & 35,04 & 44,33 & 48,12 \\ \hline
	\textit{\begin{tabular}[c]{@{}l@{}}Durchschn. Abweichung \\ p. Winkel (in Deg.)\end{tabular}} & 2,04 & 2,09 & 3,59 & 3,56 & 5,05 & 5,46 \\ \hline
	{\ul \textit{}} &  &  &  &  &  &  \\ \hline
	\textit{\begin{tabular}[c]{@{}l@{}}Gesamtwinkelsumme\\ (in Deg.)\end{tabular}} & 3894,53 & 3910,4 & 7354,98 & 7358,55 & 9398,12 & 10298,28 \\ \hline
	\textit{erkannte Punkte/Winkel} & 1912 & 1873 & 2050 & 2065 & 1860 & 1885 \\ \hline
	\textit{erkannte Polygone} & 147 & 145 & 209 & 210 & 212 & 214 \\ \hline
	\textit{verglichene Polygone} & 131 & 129 & 188 & 189 & 193 & 194 \\ \hline
	\textit{\begin{tabular}[c]{@{}l@{}}Zahl gleicher Objekte \\ in 2 Frames erkannt\end{tabular}} & 87 & 82 & 101 & 100 & 131 & 130 \\ \hline
	\textit{Prozessierungszeit (in Sek.)} & 92,9 & 91,97 & 112,21 & 102,73 & 132,66 & 102,32 \\ \hline
	\end{tabular}
	\end{table}
	}
\subsection{yolo8n}
{black video immer an, yolo8n}
\

{black video immer an, yolo8m}


\subsection{yolo8x}
{black video immer an, yolo8x}




\section{Bewertung der Testfälle im Vergleich}




\section{Ausblick}
\begin{enumerate}
	\item DCE Abbruchbedingung nicht fest implementieren sondern einen Wert einführen, der immer im Vergleich zur Ähnlichkeit des Ursprungspolygons gemessen wird. Ermöglicht eine bedarfsbezogene Vereinfachung des Polygons, individuell für jeden erkannten Umriss
	\item Implementierung der DCE; bzw. anderer Programmteile in schnellerer Programmiersprache wie C/C++; bzw. hardwarenah(er)
\end{enumerate}
{
	Wenn das Ergebnis der Anwendung von DCE für Objekttracking erfolgreich ist und die getrackten Objekte erkennbar bleiben, kann eine hardwarenahe Programmierung erfolgen. Diese könnte in C oder C++ gemacht werden, um schnellere Ergebnisse liefern zu können, da die Prozessierungsgeschwindigkeit von Python begrenzt ist.\newline
Durch die hardwarenahe Implementierung der DCE könnte eine Anonymisierung direkt am Aufzeichnungsort stattfinden. Auch ein vereinfachtes Objekttracking ist durch eine erfolgreiche Anwendung der DCE möglich.


}



