%!TEX root = ../thesis.tex
\chapter{Verwendete Technologien}

\section{Python} 
{  \label{sec:Python}
	Die Implementierung erfolgt in Python, da diese Programmiersprache weitverbreitet ist und eine einfache Einbindung weiterer Bibliotheken erlaubt \citep{Millman2011}. Durch diese leichte Erweiterungsmöglichkeit ergibt sich die Möglichkeit komplexen Programmcode zu schreiben, welcher den Rahmen dieser Arbeit nicht überschreitet. \\
	Python wurde am 14. Februar 2009 in der Version 3.0 veröffentlicht \citep{Rossum2009}. Diese Programmiersprache bietet Vorteile durch die einfache Syntax und die Unterstützung der Einbindung diverser externen Bibliotheken \citep{Marowka2018}. \\
	In dieser Arbeit wird Anaconda als Installationsumgebung und Visual Studio Code als Entwicklungsumgebung genutzt. }

\section{Externe Bibliotheken, Algorithmen und Implementierungen}
		\subsection{GeoPandas}
		{ \label{subsec:Geopandas}
			Das Geopandas Project wurde 2013 von Kelsey Jordahl gegründet. Version 0.1.0 ist im Juli 2014 veröffentlicht worden. Es ist ein Open-Source Projekt um die Unterstützung von geographischen Daten zu Pandas Objekten hinzuzufügen. \citep{kelsey_jordahl_2020_3946761}. Pandas ist eine Bibliothek zur Datenmanipulation- und -analyse \citep{reback2020pandas}.  \\
			Geopandas wird für diese Arbeit als geeignet betrachtet, weil es sich bei Polygonen in Videos um räumliche Daten handelt, die im Laufe der Arbeit mit DCE manipuliert werden.
		}
		\subsection{NumPy und Quicksort}
		{ \label{subsec:NumPy}
		NumPy ist ein Open-Source Projekt, welches 2005 gegründet wurde, um numerische Operationen in Python zu ermöglichen \citep{numpy_about}. Die aktuelle Version 1.25.0 wurde am 17.06.2023 veröffentlicht \citep{numpy_main_web}. \\
		Diese Bibliothek bietet nicht nur mehrdimensionale Arrays, sondern auch diverse numerische Operationen an. Außerdem ist NumPy durch effiziente Implementierung sehr performant \citep{numpy_main_web}. \\

		NumPy bietet verschiedene Sortieralgorithmen an, um Arrays zu ordnen. In dieser Arbeit wird das vergleichsbasierte Sortierverfahren \glqq Quicksort\grqq{} benutzt. \\
		Quicksort wurde im Jahr 1962 von Charles Antony Richard Hoare vorgestellt \citep{Hoare1962QS}. Dieser instabile Algorithmus bietet Vorteile in der Sortiergeschwindigkeit bei großen Datenmengen, da er eine durchschnittliche und beste Komplexität von $n*log(n)$ aufweist. Die schlechteste Komplexität ist $n^2$. Die Variable $n$ steht hier für die Anzahl der zu sortierenden Elemente. \\
		Quicksort wird in dieser Arbeit verwendet, da es schnell arbeitet und die Instabilität, wegen der geringen Wahrscheinlichkeit das exakt gleiche Werte sortiert werden, nur marginale Auswirkungen auf das Ergebnis hat.
		}

		\subsection{Computer Vision 2}
		{ \label{subsec:Computer_Vision_2}
		Computer Vision 2 (CV2) ist ein Teil der 'Open Source Computer Vision' Bibliothek \citep{opencv_about}. Die aktuelle Version 4.8.0 wurde am 02.07.2023 veröffentlicht \citep{opencv_release}. \\
		Diese Bibliothek beinhaltet Algorithmen zur Bild- und Videobearbeitung. In dieser Arbeit wird diese Bibliothek zur Manipulation von Frames in Videos genutzt, die zuvor mit YOLO analysiert wurden. 
		}
		\subsection{weitere Bibliotheken}
		\subsubsection*{Timer}	
		{Die externe Bibliothek \textit{timer} wird importiert, damit innerhalb des Programmablaufes Timestamps (Zeitstempel) gesetzt werden können, um einzelne Schritte des Programmes zu messen. Diese Bibliothek wurde in der Version 0.2.2 am 30.08.2021 von Lucien Shui veröffentlicht \citep{Shui2021}. }
		\subsubsection*{Datetime}{
			Die externe Bibliothek \textit{datetime} wird genutzt, um am Anfang der Result Textdateien, das aktuelle Datum und die Uhrzeit in die Textdatei zu schreiben. Diese Bibliothek wurde in der Version 5.2 am 19.07.2023 von der Zope Foundation veröffentlicht \citep{Zope2023}.
		}
		\subsection{YOLOv8 Implementierung von Canu}
		{ \label{YOLOv8_canu}
		In dieser Arbeit werden die \glqq-seq\grqq{} Modelle von YOLOv8 benutzt, da diese bereits eine Segmentierung der Objekte mit deren Umrissen integriert haben. Diese Version wurde von Ultralytics entwickelt und wird als Python Bibliothek bereitgestellt. Für eine detaillierte Beschreibung s. Kap. \ref{subsec:YOLOv8_theoretic}. \\
		Die Implementierung basiert auf \citeauthor{Canu_pysource}, da dieser eine Einführung in Bildbearbeitung mit Python unter der Benutzung von YOLO gegeben hat \citep{Canu_pysource}.
		}


	
	




	

