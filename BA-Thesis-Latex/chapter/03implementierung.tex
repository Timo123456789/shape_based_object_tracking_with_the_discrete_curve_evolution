%!TEX root = ../thesis.tex
\chapter{Implementierung (LISTINGS ÜBERPRÜFEN)} 
\label{ch:implementierung}


%\section{Implementierung in Python}
{\label{implementation_in_python}} 
Im Folgenden wird die Implementierung anhand von Codebeispielen erläutert. Die angegebenen Listings sind zum einfachen Verständnis gekürzt und ohne Kommentare. Für ein Listing des gesamten Codes mit Kommentaren siehe Anhang \ref{cd:gesamt_listing}. \\
EF Version ist die Abkürzung für \glqq Every Frame\grqq{} Version (1. YOLO Variante) und RV Version ist die Abkürzung für \glqq Result\grqq{} Version (2. YOLO Variante), die direkt mit einem von YOLO generierten Objekt arbeitet. 
\section{Main File}
{ \todo{main.py und shape\_sim\_meas Listings überprüfen!!!!}
	Das Main File importiert alle Unterskripte, da es auf die Funktionen zugreifen muss, um das Video zu verarbeiten und zu schreiben. Diese Unterskripte sind: 
	\begin{itemize}
		\item yolo\_every\_frame (siehe \ref{py:YOLO_every_frame})
		\item yolo\_result\_version (siehe \ref{py:YOLO_res_vers})
		\item DCE.py (siehe \ref{py:DCE})
		\item shape\_sim\_meas.py (siehe \ref{py:Shape_Sim_Meas})
	\end{itemize}
	Dazu wird die Erweiterung CV2 (siehe \ref{subsec:Computer_Vision_2}) eingelesen, um das Video aus einzelnen Frames zu generieren.
	Außerdem wird die externe Library \textit{timer} importiert, damit innerhalb des Programmablaufes Timestamps (Zeitstempel) gesetzt werden können, um einzelne Schritte des Programmes zu messen. Diese Library wurde in der Version 0.2.2 am 30.08.2021 von Lucien Shui veröffentlicht \citep{Shui2021}.  \\
	Die Main Methode besteht zu aus einer Dictionary Variable \lstinline|options|, in der alle Einstellungen für die Verarbeitung des Videos gesetzt werden. Hier werden auch die einzelnen Zeitstempel gespeichert. Ein Ausschnitt ist in Listing \ref{cd:part_of_options_var} zu sehen. \\
	In diesem wird die Lese- und Schreibpfade für das Quell- und Ergebnisvideo festgelegt, sowie der Pfad für die Textdatei, die die Timestamps enthält. In den nächsten Zeilen kann festgelegt werden, auf bis viele Punkte die von YOLO detektierten Objekte reduziert werden. Diese werden in 4 unterschiedliche Objekttypen unterteilt: Auto (Car), Motorrad (Motorcycle), LKW (Truck) und andere Objekte (other\_Object). \\
	Des Weiteren wird das genutzte YOLO Modell definiert und festgesetzt, ob nur die vereinfachten Umrisse der erkannten Objekte ausgegeben werden. Hier kann die Ausgabe der Labels gesteuert werden, diese beinhalten die Information, was für eine Klasse, bzw. Objekt, detektiert wurde und wie hoch der Confidence Score ist. \\
	Standardmäßig ist die Version des Codes ausgewählt, die das Video erst vollständig von YOLO analysieren lässt, dies lässt sich mit der \lstinline|yolo_every_frame| Boolean umstellen. Die letzte Boolean beschreibt, ob Zeitstempel gesetzt werden sollen. Dieses Dictionary hat noch weitere Einträge, die jedoch lediglich der Verwaltung der verschiedenen Zeitstempel und weiterer Messwerte dienen. \\
	\lstinputlisting[basicstyle=\ttfamily\scriptsize, linerange={18,19-22,28-31,33-36,38,54-61}, caption={Ausschnitt aus der \protect\lstinline|options| Variable in  Main.py}, label = {cd:part_of_options_var}]{../Code/main.py}

	Im weiteren Verlauf wird dann ausgewählte Version des Codes gestartet und beim Abschluss das Video und die Textdatei in die entsprechenden Pfade geschrieben.
}



\section{1. YOLO Variante (EF Version)} {
	\label{py:YOLO_every_frame}
	In dieser Variante wird das Video in einzelne Frames mit CV2 zerlegt, auf die dann jeweils der ausgewählte YOLO Algorithmus angewendet wird.
	Es wird über alle Frames des Videos iteriert, in der das jeweilige Frame aus dem Video extrahiert wird und dann an die Methode übergeben wird, die YOLO anwendet.
	\lstinputlisting[basicstyle=\ttfamily\scriptsize,linerange={22,24,26,27,29,31,37,42}, caption={Ausschnitt aus yolo\_every\_frame.py}, label = {cd:part_of_yolo_every_frame.py}]{../Code/YOLO/yolo_every_frame.py}
	Dies geschieht indem zuerst die Gesamtanzahl der Frames in der \lstinline|framecounter| Variablen gespeichert wird, welche den Iterator für die Schleife limitiert (siehe Listing \ref{cd:part_of_yolo_every_frame.py}). \\ 
	In der Schleife wird das jeweilige Frame an der \lstinline|i|-ten Stelle als Bilddatei in der \lstinline|img| Variablen gespeichert. Dieses wird dann im nächsten Schritt der Funktion übergeben, die das Bild mit YOLO analysiert und zurückgibt. Ein vorher festgelegtes Array speichert dann alle analysierten Bilder. \\ 
	Wenn die Schleife terminiert, wird das Array aus Bildern zu einem Video zusammengesetzt und gespeichert. \\

	
	\lstinputlisting[basicstyle=\ttfamily\scriptsize,linerange={59,64,71,73,76,83,84,86,87,93-97}, caption={Ausschnitt aus der \protect\lstinline|run_yolo| Funktion in yolo\_every\_frame.py}, label = {cd:run_yolo_func_in_yolo_every_frame.py}]{../Code/YOLO/yolo_every_frame.py}
	
	Ausschnitte der \lstinline|run_yolo| Funktion sind in Listing \ref{cd:run_yolo_func_in_yolo_every_frame.py} zu sehen. Zuerst wird hier das YOLO Modell festgelegt. Danach wird der Frame von YOLO analysiert, welches in der yolo\_segmentation.py erfolgt. Hier wird ein Objekt zurückgeben, welches die Boundingboxen der detektierten Objekte, die jeweiligen Klassen, die segmentierten Umrisse und den jeweiligen Confidence Score enthält. \\
	Dieses Objekt wird in der darauffolgenden Schleife durchlaufen. Hier werden zunächst die Koordinaten der jeweiligen Boundingbox gesetzt und im nächsten Schritt werden die Umrisse mit der DCE vereinfacht (für eine genauere Beschreibung in der Theorie siehe Kap. \ref{sec:Discrete Curve Evolution} und im Code siehe Kap. \ref{py:DCE}). Mit CV2 werden danach die Boundingbox gezeichnet und die Umrisse der Polygone gezeichnet. Für den Fall, dass die \lstinline|write_Labels| Variable im \lstinline|options| Dictionary auf True gesetzt ist, werden im nächsten Schritt der Confidence Score und die Class ID an die Boundingbox geschrieben. \\
	Zur Evaluation werden im nächsten Schritt die Winkelsummen für jedes Polygon summiert und in einem Array im  \lstinline|options| Dictionary gespeichert. \\
	Wenn alle Frames des Videos durchlaufen wurden, werden die einzelnen Frames wieder zu einem Video zusammengesetzt und statistische Auswertungen (siehe Kap. \ref{py:Shape_Sim_Meas}) durchgeführt. Damit ist dieser Abschnitt des Programmes abgeschlossen.

	\lstinputlisting[basicstyle=\ttfamily\scriptsize,linerange={13,22-24,36,38-41,44,46,48,50,51}, caption={Ausschnitt aus  yolo\_segementation.py}, label = {cd:yolo_in_yolo_segmentation.py}]{../Code/YOLO/yolo_segmentation.py}
	YOLO\_segmenation.py basiert auf einer Entwicklung von \citeauthor{Canu_pysource} \citep{Canu_pysource} und beinhaltet einige Abänderungen. Der Code auf den sich im Folgenden bezogen wird, ist in Listing \ref{cd:yolo_in_yolo_segmentation.py} zu sehen. Hier wird nach der Initialisierung des YOLO Modells die Detektionsfunktion ausgeführt. Diese beinhaltet die Prediction in Zeile 6 und eine IF Abfrage, wenn keine Objekte von YOLO erkannt wurden. Wenn Objekte erkannt wurden, werden deren Boundingboxen, ClassIDs, Umrisse und Confidence Scores zurückgeben. 

	

	}

\section{2. YOLO Variante (RV Version)}{
	\label{py:YOLO_res_vers}
	Diese Variante des Codes analysiert das Video direkt am Anfang mit YOLO. Dies bietet den Vorteil, dass durch die effiziente Implementierung von YOLO die Gesamtdauer des Programmes verringert wird. Die Anwendung von YOLO findet direkt in main.py statt. Dies ist in Listing \ref{cd:yolo_result_main.py} zu sehen.
	\lstinputlisting[basicstyle=\ttfamily\scriptsize,linerange={87,93,95,99,103,106,108}, caption={Ausschnitt aus \protect\lstinline|run_yolo_result_version| in main.py}, label = {cd:yolo_result_main.py}]{../Code/main.py}
	Da hier das gesamte Video in dem von YOLO generierten \lstinline|results| Objekt gespeichert wird, benötigt die Funktion, die das Video verändert nur dieses Objekt und das \lstinline|options| Dictionary. \\
	

	Die \lstinline|get_outline_for_every_object| Funktion (siehe Listing \ref{cd:yolo_result_get_outline_for_every_object.py}) läuft folgendermaßen ab. \\
	Die Schleife iteriert über die Gesamtanzahl der Frames im Video. Durch die IF Abfrage wird der Fall abgefangen, dass kein Objekt im Frame erkannt wurde. Wenn ein Objekt erkannt wurde, werden mit der \lstinline|get_data| Funktion die Daten der detektierten Objekte aus dem Frame exportiert. Ansonsten wird das Frame ohne Veränderung überschrieben, wenn das Video im \lstinline|options| Dictionary nicht auf schwarz gesetzt wurde. Die \lstinline|get_data| Funktion basiert auf \citeauthor{Canu_pysource} \citep{Canu_pysource} und ist ähnlich zu der in Kap. \ref{py:YOLO_every_frame} aufgebaut. \\
	Diese Daten werden in einzelne Variablen abgespeichert. Danach wird auf die Umrisse die DCE (siehe Kap. \ref{py:DCE}) angewendet. Mit CV2 werden die vereinfachten Umrisse dann in das Frame gespeichert, indem das gerade analysierte Frame überschrieben wird. \\
	\lstinputlisting[basicstyle=\ttfamily\scriptsize,linerange={11,18-20,23-25,27-30,33,39,41,42}, caption={Ausschnitt 1 aus \protect\lstinline|get_outline_for_every_object| Funktion in yolo\_result\_version.py}, label = {cd:yolo_result_get_outline_for_every_object.py}]{../Code/YOLO/yolo_result_version.py}
	Die zweite FOR Schleife (siehe Listing \ref{cd:yolo_result_get_outline_for_every_object_second.py}) zeichnet für jedes erkannte Objekt im Frame die Boundingbox und die Labels, falls dies im \lstinline|options| Dictionary gesetzt wurde. Außerdem wird noch die Gesamtsumme der Winkel  zur späteren statistischen Auswertung für jedes Polygon berechnet.
	\lstinputlisting[basicstyle=\ttfamily\scriptsize,firstnumber = 16,linerange={44-46,48,49,52,64,66,67,69,70,73}, caption={Ausschnitt 2 aus \protect\lstinline|get_outline_for_every_object| Funktion in yolo\_result\_version.py}, label = {cd:yolo_result_get_outline_for_every_object_second.py}]{../Code/YOLO/yolo_result_version.py}
	Wenn die Schleifen durchgelaufen sind, wird das veränderte \lstinline|result| Objekt und das \lstinline|options| Dictionary zurückgeben. Danach wird das Video an dem angegebenen Pfad gespeichert. Es werden statistische Auswertungen (siehe \ref{py:Shape_Sim_Meas}) durchgeführt und die Timestamps gespeichert. Damit ist dieser Teil des  Programmes abgeschlossen. \\
	}

\section{Discrete Curve Evolution}{
	\label{py:DCE}
	\todo{dce listings checken}
	Die folgende Implementierung basiert auf \citeauthor{Barkowsky2000} \citep{Barkowsky2000} (siehe Kap. \ref{sec:Discrete Curve Evolution}). Die DCE wird durch ähnlich implementierte Funktionen in beiden YOLO Implementierungen gestartet. Als Beispiel wird hier die Implementierung aus yolo\_result\_version.py genutzt. Diese ist in Listing \ref{cd:yolo_result_run_DCE.py} zu sehen. \\
	Hier wird anhand der erkannten Klassen Auto, Motorrad und LKW die Methode zur Polygonvereinfachung mit einer festen Punktgrenze ausgeführt. Dies passiert auch, falls das detektierte Objekt nicht zu diesen drei Klassen gehört. \\ 
	\lstinputlisting[basicstyle=\ttfamily\scriptsize,linerange={78,89-97}, caption={Ausschnitt aus \protect\lstinline|run_DCE| Funktion in yolo\_result\_version.py}, label = {cd:yolo_result_run_DCE.py}]{../Code/YOLO/yolo_result_version.py}
	Zurückgegeben wird ein Array, welches die vereinfachten Umrisse von jedem Objekt enthält, da das Ursprungspolygon immer mit dem vereinfachten Polygon überschrieben wird. \todo{klare Benennung! entweder Umriss, oder Polygon}\\

	\lstinputlisting[basicstyle=\ttfamily\scriptsize,linerange={92,93,96-98}, caption={Ausschnitt 1 aus \protect\lstinline|simplify_polygon| Funktion in DCE.py}, label = {cd:DCE_simplify_polygon_A1.py}]{../Code/DCE/DCE.py}

	Nun wird der Prozess innerhalb der DCE.py Datei genauer erläutert (siehe Listing \ref{cd:DCE_simplify_polygon_A1.py}). Da die DCE in dieser Arbeit mit GeoPandas (siehe Kap. \ref{subsec:Geopandas}) implementiert wurde, muss das Array zuerst in ein \lstinline|Geopandas.Geoseries| Objekt transformiert werden. \\
	Für dieses Objekt wird dann ein 2 dimensionales Array mit dem Index des jeweiligen Punktes und dem dazugehörigen K Wert erzeugt (siehe Listing \ref{cd:DCE_calc_k_for_all_points.py}). Hier wird mit einer IF Abfrage der Sonderfall abgefangen, dass der erste Punkt des Polygons (an Stelle 0) für die K Wert Berechnung genutzt wird.
	Das Array, welches diese Funktion zurückgibt, kann mit der Numpy internen Funktion \lstinline|argsort| aufgrund des 1. Wertes im Array sortiert werden. Dadurch gelangt der geringste K Wert mitsamt dem entsprechenden Index im Polygon an die erste Stelle Arrays. Dieses Array wird im folgenden als \lstinline|sort_arr| bezeichnet. Ein Punkt wird immer mit zwei neu berechneten K Werten ersetzt. \\

	\lstinputlisting[basicstyle=\ttfamily\scriptsize,firstnumber = 6,linerange={109,115,124,135,137,138,140,141,145,147,149,150,152,153,157,159,161,162,172,190,200,210}, caption={Ausschnitt 2 aus \protect\lstinline|simplify_polygon| Funktion in DCE.py}, label = {cd:DCE_simplify_polygon_A2.py}]{../Code/DCE/DCE.py}

	Es folgt danach eine While Schleife (siehe Listing \ref{cd:DCE_simplify_polygon_A2.py}) mit der Abbruchbedingung, dass die Länge von \lstinline|sort_arr| größer gleich der gewünschten Endpunktzahl (\lstinline|fNoP|, final Number of Points) ist. \\
	In dieser Schleife wird der Index des Punktes mit dem geringsten K Wert (\lstinline|indic|) aus dem \lstinline|sort_arr| gelesen und zur Löschung des Punktes an dieser Stelle im Polygon genutzt. Dieses um einen Punkt reduzierte Polygon überschreibt dann das ursprüngliche Polygon. In einer weiteren Variable wird dann die aktuelle Punktzahl des neuen Polygons gespeichert, um die Sonderfälle bei der Berechnung der K Werte abzufangen. \\
	
	Wenn \lstinline|indic| der aktuellen Punktanzahl im Polygon entspricht, werden als vorheriger Punkt der letzte Punkt und als aktueller Punkt der erste Punkt im Polygon genutzt. \\
	Da die Berechnung von K in dieser Implementierung drei Punkte erfordert, müssen die Fälle behandelt werden, wenn \lstinline|indic-1| oder \lstinline|indic+1| außerhalb des Polygons oder \lstinline|sort_arr| liegen. \\
	Wenn \lstinline|indic-1| den Wert 0 hat, muss der vorherige Punkt der erste Punkt im Polygon sein. Diese Funktion wird dann mit den Werten 0, als Punkt für den K berechnet wird, 1 für den nächsten Punkt, und \lstinline|NoP - 1| für den letzten Punkt im Polygon gestartet. \\
	Im Else Teil dieser Abfrage wird mit einer weiteren IF Abfrage abgefragt, ob \lstinline|indic+1 > NoP_temp| ist. Dies ist der Fall, wenn \lstinline|indic| der letzte Punkt im Polygon ist. Dann wird \lstinline|k_bef| mit den Werten \lstinline|indic-1, 0, indic| berechnet, da der letzte Punkt im Polyon betrachtet werden muss.
	Wenn dies nicht der Fall ist, kann \lstinline|indic-1| ohne Beachtung von Sonderfällen berechnet werden, indem der betrachtete Punkt (\lstinline|indic-1|), der nachfolgende Punkt (\lstinline|indic|) und der vorherige Punkt (\lstinline|indic-2|) in der Funktion genutzt wird.
	Außerhalb von diesem Else Teil existiert in der While Schleife eine weitere If Abfrage, die ebenfalls den Fall abdeckt, dass \lstinline|indic+1 > NoP_temp| ist. Dies ist auch für die Berechnung von dem K Wert des aktuellen (neuen) K Wertes an der Stelle \lstinline|indic| wichtig. Dieser Wert wird dann anhand des zweiten Punktes (\lstinline|indic|), zum ersten Punkt (0) und zu vorherigen Punkt (\lstinline|indic-1|) berechnet. Sonst wird die normale Berechnung mit den Werten \lstinline|indic, indic+1, indic-1| ausgeführt.

	
	\lstinputlisting[basicstyle=\ttfamily\scriptsize,linerange={218,220,221,229,231,232,240-244}, caption={Ausschnitt 1 aus \protect\lstinline|update_sort_array| Funktion in DCE.py}, label = {cd:DCE_update_sort_array_A1.py}]{../Code/DCE/DCE.py}

	Da die K Werte nun berechnet wurden, müssen diese im \lstinline|sort_arr| aktualisiert werden. Dies geschieht in der Funktion \lstinline|update_sort_array| (siehe Listings \ref{cd:DCE_update_sort_array_A1.py} - \ref{cd:DCE_update_sort_array_A4.py}), die nun näher erläutert wird. \\
	Die Funktion benötigt, das \lstinline|sort_arr|, den aktuellen Index \lstinline|indic|, sowie die zuvor berechneten K Werte für den vorherigen Punkt \lstinline|k_bef|, den aktuellen Punkt \lstinline|k_act| und das Polygon. \\
	\newpage
	Es werden zunächst zwei Boolean Variablen initialisiert, die anzeigen, ob ein Wert im \lstinline|sort_arr| gefunden wurde und überschrieben werden muss. Außerdem wird der erste Wert aus dem \lstinline|sort_arr| gelöscht. \todo{Zweimal den ersten Wert zu löschen ist doch eigentlich sinnfrei? Aber sonst schrumpft das sort Array nicht, weil immer zwei Elemente hinzugefügt werden und nur eins gelöscht wird...}\\
	Im nächsten Schritt wird dann mit einer Numpy Methode die entsprechende Stelle im \lstinline|sort_arr| gesucht an der die K Werte überschrieben werden müssen. Hier wird ebenfalls nach dem nachfolgenden Wert von \lstinline|indic| gesucht, der in der darauffolgenden IF Abfrage genutzt wird. Diese Abfrage fängt den Fall ab, dass \lstinline|indic| den Wert 0 hat oder der nächste Wert nicht existiert. Wenn dies der Fall ist, wird wieder der erste Wert des \lstinline|sort_arr| Arrays gelöscht, sonst wird der letzte Wert gelöscht.

	\lstinputlisting[basicstyle=\ttfamily\scriptsize,linerange={249,250,252-256,259,261-265,268-270},firstnumber=12, caption={Ausschnitt 2 aus \protect\lstinline|update_sort_array| Funktion in DCE.py}, label = {cd:DCE_update_sort_array_A2.py}]{../Code/DCE/DCE.py}

	Wenn weiterhin \lstinline|indic != 0| gilt, wird mit zwei aufeinanderfolgenden IF Abfragen der Wert der gesuchten Indizes im \lstinline|sort_arr| Array abgefragt. Da diese immer aus einem 2 dimensionales Array bestehen, kann mit einer Größenabfrage abgefragt werden, ob ein Wert gefunden wurde. Wenn die Größe den Wert 0 hat, bzw. kein Wert gefunden wurde, wird ein Numpy Array mit dem entsprechenden Index und K Wert erzeugt, welches anschließend mit dem \lstinline|sort_arr| verbunden wird. Dies passiert, damit immer für jeden Punkt im Polygon ein K Wert vorhanden ist, wenn dieser vorher entfernt wurde.


	\lstinputlisting[firstnumber=28, basicstyle=\ttfamily\scriptsize,linerange={273-280}, caption={Ausschnitt 3 aus \protect\lstinline|update_sort_array| Funktion in DCE.py}, label = {cd:DCE_update_sort_array_A3.py}]{../Code/DCE/DCE.py}

	In Ausschnitt 3 (siehe Listing \ref{cd:DCE_update_sort_array_A3.py}) werden die K Werte im \lstinline|sort_arr| aktualisiert. Auch hier wird zunächst der Sonderfall betrachtet, wenn \lstinline|Indic = 0| ist, da nun der letzte Punkt (als vorheriger) und erste Punkt (als aktueller) im Array aktualisiert werden muss. Sonst wird abgefragt, ob ein Wert an das \lstinline|sort_arr| angehängt wurde, was ein Überschreiben des Wertes überflüssig macht. Wenn dies nicht der Fall ist, wird das entsprechende Element im Array mit dem neuberechneten K Wert überschrieben.

	\lstinputlisting[firstnumber=36, basicstyle=\ttfamily\scriptsize,linerange={289,292,293,304,306}, caption={Ausschnitt 4 aus \protect\lstinline|update_sort_array| Funktion in DCE.py}, label = {cd:DCE_update_sort_array_A4.py}]{../Code/DCE/DCE.py}

	Zuletzt werden die Indizes des \lstinline|sort_arr| neu nummeriert, da es sonst beim Löschen ein Wert außerhalb des Polygons angesteuert wird. \todo{ich weiß auch noch nicht so genau, warum das überhaupt der Fall ist. Er löscht den aktuellen Wert, aktualisiert den vorherigen und fügt den neuberechneten hinzu -> eigentlich muss es auch ohne Neunummerierung schrumpfen} Im letzten Schritt wird das Array wieder nach dem geringsten K Wert sortiert und daraufhin zurückgeben. \\
	Damit ist die While Schleife aus Listing \ref{cd:DCE_simplify_polygon_A2.py} fast beendet. Es wird in dieser die aktuelle Punktzahl mit der gewünschten verglichen und das Polygon wird im Erfolgsfall zurückgeben. Ansonsten beginnt die Schleife von vorne. \\
	Nun wird die Berechnung des K Wertes beschrieben.

	\lstinputlisting[basicstyle=\ttfamily\scriptsize,linerange={358-362,364,367-369,373,376}, caption={Ausschnitt aus \protect\lstinline|calc_k_for_all_points| Funktion in DCE.py}, label = {cd:DCE_calc_k_for_all_points.py}]{../Code/DCE/DCE.py}

	Der K Wert wird in der folgenden Funktion \lstinline|calc_k_with_points| berechnet. Diese Methode basiert auf der Formel \ref{Equ_K_Bark} ( nach \citet{Latecki1999a})und ist dahingehend abgeändert worden, dass diese nicht mit Liniensegmenten, sondern mit drei Punkten berechnet werden kann. Die abgeänderte Formel lautet:
	\begin{equation}
		K(p,S_1,S_2) = \frac{\beta(p, S_1, S_2)l(p, S_1)l(p, S_2)}{l(p, S_1) + l(p, S_2)} 
		\label{equ_K_DCE_points}
	\end{equation}
	Im Programmcode ist dies folgendermaßen implementiert (siehe Listing \ref{cd:DCE_calc_k_with_points.py}). Es wird zunächst der Winkel $\beta$ berechnet, indem alle drei Punkte und das Polygon übergeben werden. Danach werden die beiden Distanzen zwischen $p, S_1$ und $p, S_2$ berechnet. Die Winkelberechnungsfunktion arbeitet mit NumPy und die Distanzberechnungsfunktion basiert auf Funktionen, die GeoPandas bietet.
	\lstinputlisting[basicstyle=\ttfamily\scriptsize,linerange={406,418-420,422,423}, caption={Ausschnitt aus \protect\lstinline|calc_k_with_points| Funktion in DCE.py}, label = {cd:DCE_calc_k_with_points.py}]{../Code/DCE/DCE.py}
	Diese drei Werte werden dann analog zur Formel \ref{equ_K_DCE_points} zum Wert K berechnet. Als letzten Schritt wird dieser Wert und der Winkel als Array zurückgegeben.
	
	In Listing \ref{cd:DCE_angle_and_distance.py} ist zu sehen, dass die Winkelberechnungsfunktion mit dem Arcustangens und $\pi$ arbeitet. Dies basiert auf der Dokumentation von NumPy \citep{numpy_angle}. Da die Berechnung ein Array ausgibt, müssen alle Elemente aufsummiert werden, um einen Wert zu erhalten. Dieser Wert wird dann in Radiant umgerechnet und zurückgeben. \\
	Die Distanzberechnung erfolgt mit einer Funktion, die GeoPandas bei \lstinline|Geopanda.Geoseries| Objekt bietet. Diese errechnet die Distanz zwischen den Punkten \lstinline|p1| und \lstinline|p2|. Danach wird diese zurückgegeben.
	\lstinputlisting[basicstyle=\ttfamily\scriptsize,linerange={545,557-559,569,570,572,573,577,596,605-608}, caption={Ausschnitte aus \protect\lstinline|get_angle_two_lines| und \protect\lstinline|get_distance_between_two_points| Funktionen in DCE.py}, label = {cd:DCE_angle_and_distance.py}]{../Code/DCE/DCE.py}
	Damit sind alle wesentlichen Funktionen der DCE.py Datei beschrieben. Für weitere Dokumentation und den gesamten Code mit Kommentaren ist dieser im Anhang \ref{cd:listing_DCE.py} zu sehen.
}


\section{Shape Similarity Measure}{
	\label{py:Shape_Sim_Meas}
	Beide Versionen der YOLO Implementierung erzeugen ein mehrdimensionales Array, welches für jedes Frame jedes Polygon mit Gesamtwinkelsumme, Class ID und Umriss enthält. Dieses mehrdimensionale Array wird ausgewertet, um einzuschätzen, ob die DCE zum Objekttracking geeignet ist. 

	Die Ergebnisse werden in einem weiteren Dictionary gespeichert, welches als letzten Schritt in eine Textdatei geschrieben wird, um alle Ergebnisse und Einstellungen zentral zu speichern. In Listing \ref{cd:SSM_main.py} ist zu sehen, in welcher Reihenfolge die Analysemethoden aufgerufen werden. 
	\lstinputlisting[basicstyle=\ttfamily\scriptsize,linerange={4,11-17}, caption={Ausschnitt aus \protect\lstinline|calc_shape_similarity| Funktion in shape\_sim\_meas.py}, label = {cd:SSM_main.py}]{../Code/Shape_Similiarity/shape_sim_meas.py}
	Nachdem die Zeitstempel berechnet und im Results Dictionary gespeichert wurden, wird die erste Analysemethode aufgerufen. Dies geschieht im Detail folgendermaßen.

	Die folgende Methode \lstinline|calc_shape_similarity_compare_polygons| vergleicht die Winkelsummen jedes einzelnen Polygon mit dem ähnlichen Polygon im nächsten Frame. Dies geschieht, indem bei der Detektion der Objekte nach dem Vereinfachen mit DCE ein mehrdimensionales Array mit den Daten [Framenummer, Polygonnummer, Gesamtwinkelsumme (im Polygon), ClassID, [Umrisskoordinaten]] angelegt wird, welches dann zu einem weiteren Array hinzugefügt wird. Dadurch ist jedes Polygon in jedem Frame einzeln identifzier- und vergleichbar. 
	Die Auswertung dieses mehrdimensionalen Arrays erfolgt ähnlich zu Listing \ref{cd:shape_sim_meas_compare_poly.py}.\\

	Die Schleife iteriert über alle Frames. Innerhalb dieser wird abgefragt, ob im nächsten Frame mehr Objekte detektiert wurden, als im betrachteten Frame. Dies ist wichtig, damit die nächste Schleife nur über so viele Polygone iteriert, wie das Frame mit der geringeren Anzahl hat. Wenn das nächste Frame jedoch weniger Polygone als das betrachtete Frame hat, dürfen nur so viele Polygone betrachtet werden, wie im nächsten Frame vorhanden sind. \\
	\lstinputlisting[basicstyle=\ttfamily\scriptsize,linerange={22,30,31,33-36,38-44}, caption={Ausschnitt 1 aus \protect\lstinline|calc_shape_similarity_compare_polygons| Funktion in sha\-pe\_sim\_me\-as.py}, label = {cd:shape_sim_meas_compare_poly.py}]{../Code/Shape_Similiarity/shape_sim_meas.py}
	Die integrierte FOR Schleife (siehe Listing \ref{cd:shape_sim_meas_compare_poly_second.py}) vergleicht die Winkelsummen der einzelnen Polygone. Dies wird dadurch ermöglicht, dass die temporäre Variable (\lstinline|temp|) nur dann berechnet wird, wenn das Polygon im betracheten Frame an der gleichen Stelle im Array steht, wie das Polygon im nächsten Frame. 
	\lstinputlisting[basicstyle=\ttfamily\scriptsize,firstnumber=15, linerange={45-51}, caption={Ausschnitt 2 aus \protect\lstinline|calc_shape_similarity_compare_polygons| Funktion in sha\-pe\_sim\_me\-as.py}, label = {cd:shape_sim_meas_compare_poly_second.py}]{../Code/Shape_Similiarity/shape_sim_meas.py}
	Diese temporäre Variable muss immer positiv sein, was durch die IF Abfrage sichergestellt ist, da dies sonst die Differenzvariable verfälscht. Danach wird die Differenzvariable mit der temporären Variable addiert. Es wird nun die Gesamtanzahl der Punkte berechnet, welche identisch mit der Anzahl der berechneten Winkel ist. Zuletzt wird diese mit der Differenzvariable im \lstinline|options| Dictionary gespeichert. 
	\lstinputlisting[basicstyle=\ttfamily\scriptsize,firstnumber=22, linerange={53,54,56-59,61-63}, caption={Ausschnitt 3 aus \protect\lstinline|calc_shape_similarity_compare_polygons| Funktion in sha\-pe\_sim\_me\-as.py}, label = {cd:shape_sim_meas_compare_poly_third.py}]{../Code/Shape_Similiarity/shape_sim_meas.py} 
	Die statistischen Auswertungen, werden alle im Dictionary (siehe Listing \ref{cd:shape_sim_meas_compare_poly_third.py}) in Radiant und Degree gespeichert. Hier wird  die Winkeldifferenz über alle Frames und Polygone, die durchschnittliche Winkelabweichung der pro Polygon und pro Winkel in das Dictionary geschrieben. Außerdem wird noch die Gesamtanzahl der erkannten Punkte gespeichert, die der Anzahl der Winkel entspricht, da zu jedem Punkt in den Polygonen ein Winkel berechnet wurde. \\
	
	Danach wird die Winkelabweichung mit der Methode in Listing \ref{cd:shape_sim_meas_one_frame_first.py} berechnet, welche alle Polygone der gleichen Klasse in einem Frame miteinander vergleicht. Der Messwert wird dadurch berechnet, dass über alle Frames iteriert wird.
	\lstinputlisting[basicstyle=\ttfamily\scriptsize,linerange={70,92,93,95,96,98,99,101,102,104}, caption={Ausschnitt aus \protect\lstinline|calc_shape_sim_compare_classes_in_one_frame| Funktion in sha\-pe\_sim\_me\-as.py}, label = {cd:shape_sim_meas_one_frame_first.py}]{../Code/Shape_Similiarity/shape_sim_meas.py} 
	Die Methode \lstinline|compare_polygons_in_frame| gibt ein Array zurück, welches die aufsummierten Abweichungen enthält. Dies geschieht indem mit der \lstinline|get_inidzes_and_classes| Methode die Winkelsumme des ersten Polygons jeder Klasse extrahiert wird und in der nächsten Methode \linebreak \lstinline|calc_measures_in_one_frame| diese mit allen anderen Polygonen der gleichen Klasse im Frame verglichen wird. Das Array mit den extrahierten Winkelsummen jeder Klasse wird im folgenden als Referenzklasse bezeichnet.\\
	Das Vergleichen geschieht indem, wie in Listing \ref{cd:shape_sim_meas_calc_SSM_one_frame_first.py} als Beispielausschnitt zu sehen, indem über alle Polygone im Frame iteriert wird. Wenn die ClassID, in diesem Fall 2 für Car, mit der des Polygons im Frame übereinstimmt, kann die Differenzvariable berechnet werden. 
	\lstinputlisting[basicstyle=\ttfamily\scriptsize,linerange={235,236,247-254}, caption={Ausschnitt 1 aus \protect\lstinline|calc_measure_in_one_frame| Funktion in sha\-pe\_sim\_me\-as.py}, label = {cd:shape_sim_meas_calc_SSM_one_frame_first.py}]{../Code/Shape_Similiarity/shape_sim_meas.py} Diese wird in einem Array gespeichert, welches am Ende zurückgeben wird. Analog geschieht dies für die Klassen Truck und Motorcycle. Wenn keine dieser Klassen erkannt wurde, wird folgende Ausnahme behandelt (siehe Listing \ref{cd:shape_sim_meas_calc_SSM_one_frame_second.py}). 
	\lstinputlisting[ basicstyle=\ttfamily\scriptsize, firstnumber=11, linerange={267-275}, caption={Ausschnitt 2 aus \protect\lstinline|calc_measure_in_one_frame| Funktion in sha\-pe\_sim\_me\-as.py}, label = {cd:shape_sim_meas_calc_SSM_one_frame_second.py}]{../Code/Shape_Similiarity/shape_sim_meas.py} 
	Aus dem Array mit den Referenzvariablen wird, falls vorhanden, die Winkelsumme abgerufen. Wenn diese Winkelsumme nicht vorhanden ist, wird dem Array der Differenzvariable 0 addiert um diese nicht zu ändern. Sonst wird analog zum obigen Verfahren der Messwert berechnet, nur mit der Änderung das die ClassID zu jedem Wert gespeichert wird, um die Vergleichbarkeit zu gewährleisten. \\
	Damit ist das Erstellen der Arrays und die Schleife in Listing \ref{cd:shape_sim_meas_calc_SSM_one_frame_first.py} abgeschlossen. Diese Arrays werden aufsummiert. Bei Objekten, welche nicht den drei Hauptklassen entsprachen, wird mit einer Methode, welche die berechneten Werte in einzelne Arrays für jede detektierte Klasse sortiert (ähnlich zu Bucketsort), die Winkelabweichung berechnet. Für eine nähere Beschreibung siehe Anhang \ref{cd:listing_shape_sim_meas.py} ab Zeile 125. \todo{genauer beschreiben?} Eine Berechnung ist falsch, wenn nur ein Element einer Klasse pro Frame detektiert wurde, da die Winkeldifferenz dann immer 0 beträgt. Falls dies der Fall ist, wird dies im Results Dictionary für die jeweilige Klasse mitgeteilt.
	Wenn alle Werte in das Result Dictionary geschrieben wurden, wird dieses als Textdatei abgespeichert und das Programm ist vollständig abgeschlossen.

}







