%!TEX root = ../thesis.tex
\chapter{Umsetzung} \todo{Titel ist noch doof.. Aber Implementierung greift \Ref{implementation_in_python} vor}
\label{ch:implementierung}
{Im Folgenden wird die Implementierung beschrieben. Diese erfolgte in der Programmiersprache Python, da diese weitverbreitet ist und eine einfache Einbindung weiterer Bibliotheken erlaubt \citep{Millman2011}. Durch diese leichte Erweiterungsmöglichkeit ergibt sich die Möglichkeit komplexen Programmcode zu schreiben, welcher den Rahmen dieser Arbeit nicht überschreitet.

}
%\blindtext

%\section{Verwendete Technologien}
\section{Python}

\section{Externe Bibliotheken}
	\subsection{GeoPandas}
	\subsection{Numpy}
	\subsection{Computer Vision 2}
	\subsection{YOLOv8\label{YOLOv8_Ultralytics}}
	{Im Januar 2023 wurde von der Firma Ultralytics, die auch YOLOv5 entwickelt hat, YOLOv8 veröffentlicht. Diese Version beinhaltet 5 verschiedene Modelle, die mit unterschiedlich großen Datensätzen trainiert wurden  \citep{Terven2023}: 
	\begin{itemize}
		\item YOLOv8n (nano)
		\item YOLOv8s (small)
		\item YOLOv8m (medium)
		\item YOLOv8l (large)
		\item YOLOv8x (extra large)
	\end{itemize}
	Ein Vorteil dieser YOLO Version ist, dass verschiedene Varianten für Objektdetektion, -segmentierung, -verfolgunng und -klassifizierung existieren. In dieser Arbeit wird hauptsächlich YOLOv8n-seg verwendet, welches bereits eine Segmentierung der Umrisse der detektierten Objekte integriert hat.
	}
\section{Meta-Ablauf im Programm}{}

\section{Implementierung in Python}{\label{implementation_in_python}}
\subsection{Main File}
\subsection{Discrete Curve Evolution}
\subsection{1. YOLO Variante}
\subsection{2. YOLO Variante}
\subsection{Shape Similarity Measure}
\lstinputlisting[firstline = 10, lastline=20]{../Code/main.py}
\lstinputlisting[firstline = 18, lastline = 34, firstnumber = last, caption = {test}]{../Code/main.py}


	
Ein paar Zitate \cite{Hartley2004} und \cite{Bishop2006} Hier kommt noch ein Zitat 
\cite{DorrChristopherH.2015SSBo}
\blindmathpaper
% \Blindtext


Ein paar Zitate \cite{Hartley2004} und \cite{Bishop2006} Hier kommt noch ein Zitat 
\cite{DorrChristopherH.2015SSBo}

\begin{table}[ht]
	\centering
	\begin{tabular}{c|c|c}
		a & b & c \\ \hline
		1 & 2 & 3 \\
		1 & 2 & 3 \\
		1 & 2 & 3
	\end{tabular}
	\caption{Eine Tabelle}
\end{table}

\begin{figure}[ht]
	\centering
	\includegraphics[width=0.3\textwidth]{example-image-a}
	\caption{Eine Unterschrift}
\end{figure}

\begin{figure}[ht]
	\centering
	\begin{subfigure}[b]{0.45\textwidth}
		\includegraphics[width=\textwidth]{example-image-a}
		\caption{Eine Unterschrift}
	\end{subfigure} \hfill
	\begin{subfigure}[b]{0.45\textwidth}
		\includegraphics[width=\textwidth]{example-image-b}
		\caption{Noch eine Unterschrift}
	\end{subfigure}
	\caption{Mehr Unterschriften}
\end{figure}



