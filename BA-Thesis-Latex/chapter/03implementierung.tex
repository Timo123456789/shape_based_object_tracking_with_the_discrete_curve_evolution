%!TEX root = ../thesis.tex
\chapter{Umsetzung} \todo{Titel ist noch doof.. Aber Implementierung greift \Ref{implementation_in_python} vor}
\label{ch:implementierung}
{Im Folgenden wird die Implementierung beschrieben. Diese erfolgte in der Programmiersprache Python, da diese weitverbreitet ist und eine einfache Einbindung weiterer Bibliotheken erlaubt \citep{Millman2011}. Durch diese leichte Erweiterungsmöglichkeit ergibt sich die Möglichkeit komplexen Programmcode zu schreiben, welcher den Rahmen dieser Arbeit nicht überschreitet.

}
%\blindtext

%\section{Verwendete Technologien}
\section{Python}
{Python wurde am 14. Februar 2009 in der Version 3.0 veröffentlicht \citep{Rossum2009}. Diese Programmiersprache bietet Vorteile durch die einfache Syntax und die Unterstützung der Einbindung diverser externen Bibliotheken \citep{Marowka2018}. \\
In dieser Arbeit wird Anaconda als Entwicklungs- und Installationsumgebung genutzt. Im Folgenden werden die in dieser Arbeit genutzten Bibliotheken näher erläutert.  \todo{ vielleicht noch weiter ausführen?}}

\section{Externe Bibliotheken}
	\subsection{GeoPandas}
	{Das Geopandas Project wurde 2013 von Kelsey Jordahl gegründet. Version 0.1.0 ist im Juli 2014 veröffentlicht worden . Es ist ein Open-Source Projekt um die Unterstützung von geographischen Daten zu Pandas Objekten hinzuzufügen. \citep{kelsey_jordahl_2020_3946761}. Pandas ist eine Bibliothek zur Datenmanipulation- und -analyse \citep{reback2020pandas}.  \\
	Geopandas wird für diese Arbeit als geeignet betrachtet, weil es sich bei Polygonen in Videos um räumliche Daten handelt, die im Laufe der Arbeit mit DCE manipuliert werden.
	}
	\subsection{Numpy}
	{
	NumPy ist ein Open-Source Projekt, welches 2005 gegründet wurde, um numerische Operationen in Python zu ermöglichen \citep{numpy_about}. Die aktuelle Version 1.25.0 wurde am 17.06.2023 veröffentlicht \citep{numpy_main_web}. \\
	Diese Bibliothek bietet nicht nur mehrdimensionale Arrays sondern auch diveres numerische Operationen an. Außerdem ist NumPy durch effiziente Implementierung sehr performant \citep{numpy_main_web}. \\
	}
	\subsection{Computer Vision 2}
	{
	Computer Vision 2 (CV2) ist ein Teil der 'Open Source Computer Vision' Bibliothek \citep{opencv_about}. Die aktuelle Version 4.8.0 wurde am 02.07.2023 veröffentlicht \citep{opencv_release}. \\
	Diese Bibliothek beinhaltet Algorithmen zur Bild- und Videobearbeitung. In dieser Arbeit wird diese Bibliothek zur Manipulation von Frames in Videos genutzt, die zuvor mit YOLO analysiert wurden. 
	}
	\subsection{YOLOv8\label{YOLOv8_Ultralytics}}
	{
	 Es wird in dieser Arbeit das Modell YOLOv8n-seq benutzt, da dies bereits eine Segmentierung der Objekte integriert hat. Für eine detallierte Beschreibung siehe \ref{subsec:YOLOv8_theoretic}. \\
	 Die Implementierung basiert auf \citeauthor{Canu_pysource}, da dieser eine Einführung in Bildbearbeitung mit Python unter der Benutzung von YOLO gegeben hat \citep{Canu_pysource}.
	}
\section{Meta-Ablauf im Programm}{
	
}

\section{Implementierung in Python}{\label{implementation_in_python}}
\subsection{Main File}
\subsection{Discrete Curve Evolution}
\subsection{1. YOLO Variante}
\subsection{2. YOLO Variante}
\subsection{Shape Similarity Measure}
\lstinputlisting[firstline = 10, lastline=20]{../Code/main.py}
\lstinputlisting[firstline = 18, lastline = 34, firstnumber = last, caption = {test}]{../Code/main.py}


	
Ein paar Zitate \cite{Hartley2004} und \cite{Bishop2006} Hier kommt noch ein Zitat 
\cite{DorrChristopherH.2015SSBo}
\blindmathpaper
% \Blindtext


Ein paar Zitate \cite{Hartley2004} und \cite{Bishop2006} Hier kommt noch ein Zitat 
\cite{DorrChristopherH.2015SSBo}

\begin{table}[ht]
	\centering
	\begin{tabular}{c|c|c}
		a & b & c \\ \hline
		1 & 2 & 3 \\
		1 & 2 & 3 \\
		1 & 2 & 3
	\end{tabular}
	\caption{Eine Tabelle}
\end{table}

\begin{figure}[ht]
	\centering
	\includegraphics[width=0.3\textwidth]{example-image-a}
	\caption{Eine Unterschrift}
\end{figure}

\begin{figure}[ht]
	\centering
	\begin{subfigure}[b]{0.45\textwidth}
		\includegraphics[width=\textwidth]{example-image-a}
		\caption{Eine Unterschrift}
	\end{subfigure} \hfill
	\begin{subfigure}[b]{0.45\textwidth}
		\includegraphics[width=\textwidth]{example-image-b}
		\caption{Noch eine Unterschrift}
	\end{subfigure}
	\caption{Mehr Unterschriften}
\end{figure}



