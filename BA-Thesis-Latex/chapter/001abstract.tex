%!TEX root = ../thesis.tex
\begin{abstract}
\section*{Abstract}
\todo{muss noch ein bisschen weiter ausgebaut werden}
In der folgenden Arbeit wird ein Ansatz zum Objekttracking mit YOLO und der Discrete Curve Evolution (DCE) betrachtet. Dieses Objekttracking kann mit einem Formähnlichkeitsmaß für jedes Polygon bewertet werden. Es wird eine prototypische Implementierung beschrieben, die auf der Programmiersprache Python basiert. Die Evaluation des Ansatzes erfolgt an mehreren Testvideos mit verschiedenen YOLO-Modellen. Im Rahmen dieser Arbeit zeigt sich, dass das Formähnlichkeitsmaß eine geringe Abweichung von ca. 5 Grad pro Winkel aufweist, was ein Objekttracking ermöglicht. Damit kann die DCE zu einem Objekttracking beitragen.

\end{abstract}
