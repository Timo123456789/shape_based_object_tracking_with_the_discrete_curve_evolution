%!TEX root = ../thesis.tex
\begin{abstract}
\section*{Abstract}
\todo{muss noch ein bisschen weiter ausgebaut werden}
In dieser Arbeit wird ein Ansatz zu formbasiertem Objekttracking mit der Discrete Curve Evolution (DCE) vorgestellt. Zu Detektion der Objekte wird maschinelles Lernen namens YOLO verwendet. Das Objekttracking kann mit einem Formähnlichkeitsmaß für jedes Polygon bewertet werden. Es wird eine prototypische Implementierung beschrieben, die auf der Programmiersprache Python basiert. Die Evaluation des Ansatzes erfolgt an mehreren Testvideos mit verschiedenen YOLO-Modellen. Im Rahmen dieser Arbeit zeigt sich, dass das erkannten und vereinfachten Polygone geringe Abweichungen aufweisen, wodurch ein Objekttracking mit der DCE ermöglicht wird. 

\end{abstract}
